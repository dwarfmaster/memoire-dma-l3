
\subsection{Intuition}

Après l'échec de la partie précédente, revenons aux bases. Qu'est-ce qu'un CSP
sur un alphabet de relation $\Sigma$~? C'est la donnée d'un domaine fini $D$ et
de relations sur $D$ de bonne arité pour chaque relation de $\Sigma$. Si
$\Sigma = \{R_1, \dots R_n\}$, on peut poser la catégorie qui a pour objets
$\{\star, R_1, \dots R_n\}$ et commes flèches les identités, ainsi pour pour
tout $1\leq i\leq n$, autant de flèche de $\star$ vers $R_i$ que l'arité de
$R_i$. Un préfaisceau sur cette catégorie nous donne alors un CSP (et
vice-versa, par contre plusieurs préfaisceaux peuvent correspondre au même
CSP).

Pourquoi ne pas se contenter de ça ? Parce que cette catégorie n'est pas
intéressante pour travailler sur les CSP. En effet, elle ne contient comme
information que la liste des relations et leur arité, et rien d'autre. Nous la
noterons d'ailleurs $\Sigma$, puisque la donnée d'un alphabet de relation est
aussi exactement la donnée de leur liste et leurs arités. $\Sigma$ a donc la forme~:

\[\begin{tikzcd}
    & \star \arrow[ddl, shift right, swap, "f_1^1"]
            \arrow[ddl]
            \arrow[ddl, shift left, "f_{\ar(R_1)}^1"] % TODO fix
            \arrow[ddr, shift right, swap, "f_1^m"]
            \arrow[ddr]
            \arrow[ddr, shift left, "f_{\ar(R_m)}^m"]
            \\
    & & \\
    R_1 & \dots & R_m \\
\end{tikzcd}\]

C'est là que le résultat suivant entre en jeu.

\begin{prop}
    Soit $Y : \psh(\C)\rightarrow\shc(\psh(\C))$ le plongement de Yoneda sur
    $\psh(\C)$.

    Notons $y : \C\rightarrow\psh(\C)$ le plongement de Yoneda sur $\C$. Posons~:

    \[L = \left\{\begin{array}{ccc}
             \shc(\psh(\C)) & \rightarrow & \psh(\C) \\
             F              & \mapsto     & F\circ y^\text{op} \\
    \end{array}\right.\]

    Alors $L\circ Y\cong\text{Id}_{\psh(\C)}$ et
    $Y\circ L\cong\text{Id}_{\shc(\psh(\C))}$. Cela fait de $Y$ une équivalence de
    catégories.
\end{prop}

\begin{proof}
    TODO
\end{proof}

Par cet isomorphisme, $\shc(\psh(\Sigma))$ correspond aux CSP. On aimerait donc pouvoir
interpréter cette fois un préfaisceau sur $\Sigma$ comme une formule. C'est en fait assez
simple, on procède par analogie avec la façon de représenter un graphe dirigé. Un graphe
(au sens informatique) est un préfaisceau (fini) sur la catégorie $G$ suivante~:

\[\begin{tikzcd}
    v \arrow[r, bend right, "s"] \arrow[r, bend left, "t"] & e \\
\end{tikzcd}\]

Un tel préfaisceau $F : G^{\text{op}}\rightarrow\fset$ représente le graphe ayant pour
ensemble sous jacent $Fe$ et pour arrêtes $Fv$. De plus, la source d'une arrête $a\in Fv$
est $Fs(a)$ et sa destination est $Ft(a)$. On peut aussi interpréter un graphe comme
étant une formule conjonctive, dont les variables sont les points et qui possède un
unique prédicat binaire qui marque les arrêtes. On remarque que la catégorie $G$ est
exactement celle d'un alphabet avec un seul prédicat binaire. On peut donc généraliser
l'interprétation.

\begin{defi}{Formule associée à un faisceau}
    Soit $A\in\psh(\Sigma)$. On note $\phi_A$ la formule suivante~:
    
    \[\bigwedge_{1\leq i\leq m} \bigwedge_{x\in AR_i}
            R_i(Af_1^i(x), \dots Af_{\ar(R_i)}^i(x)) \]

    Avec la règle qu'une conjonction vide est $\top$

    C'est une formule conjonctive sur $\Sigma$ et dont les variables sont prises dans
    l'ensemble $A\star$.
\end{defi}

On pose alors $\C_\Sigma = \psh(\Sigma)$ notre nouvelle catégorie des formules.

\subsection{Propriétés générales de $\C_\Sigma$}

\begin{lem}
    $\C_\Sigma$ possède toutes ses limites et colimites finies.
\end{lem}

\begin{proof}
    C'est une propriété générale des catégories des faisceaux.
\end{proof}

On peut alors étudier quelques limites particulières.

\begin{lem}
    Soit $\mathbf{0}$ l'objet inital de $\C_\Sigma$. Alors $\phi_\mathbf{0}\iff\top$.
\end{lem}

\begin{proof}
    Une limite sur une catégorie de faisceau se calcule point par point. Donc $\mathbb{0}$
    est le préfaisceau qui envoie tous les objets de $\Sigma$ sur $\emptyset$ et toutes
    les flèches sur $\id_\emptyset$.

    Alors $\phi_\mathbf{0} = \bigwedge_{1\leq i\leq m}\top \iff \top$.
\end{proof}

\begin{lem}
    Soit $\mathbf{1}$ l'objet final de $\C_\Sigma$. Alors~:

    \[\phi_\mathbf{1} \iff \bigwedge_{1\leq i\leq m}R_i(x,\dots x)\]
\end{lem}

\begin{proof}
    Même démarche que pour la proposition précédente.
\end{proof}

\begin{lem}
    Soient $A$ et $B$ deux objets de $\C_\Sigma$. On note $A\wedge B$ leur coproduit
    dans $\C_\Sigma$. On a alors~:

    \[\phi_{A\wedge B} \iff \phi_A\wedge\phi_B \]
\end{lem}

\subsection{Description syntaxique des morphismes de formules}

Étudions maintenant les morphismes de formules, qui sont ici des transformations
naturelles entre faisceau. Regardons un peu plus précisément comment elles se
comportent. Soient $F,G\in\C_\psh(\Sigma)$ deux formules et $\theta:F\rightarrow F$
une transformation naturelle. Regardons comment elle doit se comporter~:

\[ \theta_\star : F\star \rightarrow G\star \]

L'existence de cette fonction nous dit que l'on renomme les variables de $F$ en des
variables de $G$, éventuellement en en contractant certaines. De plus, pour chaque
$R$ relation on a~:

\[ \theta_R : FR\rightarrow GR\]

Si on regarde comment $\phi_F$ et $\phi_G$ sont définies, cela signifie que chaque
instance de relation de $F$ est envoyée sur une instance de la même relation de $G$,
de nouveau en en contractant éventuellement certaines. Finalement, la contrainte
de naturalité est la suivante~:

\[\begin{tikzcd}
    F\star \arrow[d, "\theta_\star"] & FR_j \arrow[l, "Ff_i^j"]
                                            \arrow[d, "\theta_{R_j}"] \\
    G\star & GR_j \arrow[l, "Gf_i^j"] \\
\end{tikzcd}\]

Cela nous dit qu'après avoir renommé les variables de $F$ par $\theta_\star$, les
instance de relation sont envoyées sur des instances identiques dans $G$.

Ces différentes opérations nous rappellent un peu le calcul des séquent, et
c'est en effet le cas. Afin de le définir plus formellement, il va falloir décrire
les formules sous forme de séquents.

\begin{defi}{Séquent associé à une formule}
    Soit $\phi$ une formule conjonctive que l'on suppose mise sous la
    forme canonique $\exists x_1,\dots\exists x_n, R(\dots)\wedge R(\dots)$. On
    y associe le séquent $<\phi> = x_1,\dots x_n\vdash R(\dots), \dots, R(\dots)$.
\end{defi}

\begin{rem}
    L'habitué remarquera que la partie droite du séquent est utilisée de manière non
    canonique. En effet, elle sert usuellement à représenter une disjonction et
    l'on s'en sert ici pour représenter un conjonction. C'est lié au fait que
    les faisceaux partent de $\C_\Sigma^\text{op}$, et donc que le sens de lecture
    des règles de déduction est inversé lorsque pris par le faisceau. Ainsi, si l'on
    veut avoir des règles correspondant aux conjonctions à l'arrivée, il faut dualiser
    les règles, et cela nous donne les règles de disjonction, ce qui est cohérent avec
    l'interprétation standard des séquents.
\end{rem}

Nous rappelons maintenant les règles de déduction structurelle adaptées à nos séquents
(ce sont presque exactement celles du calcul des séquents intuitionistes).

\begin{defi}{Règles de déduction}
    Commençons par la règle de l'$\alpha$-renommage~:

    \[ \tproof{ \hypo{\Gamma, x\vdash \Delta}
         \infer1[$y\not\in\Gamma$]{\Gamma, y\vdash \Delta[x/y]} }\]

    Viennent ensuite les règles de commutation~:

    \[ \tproof{ \hypo{\Gamma, x, y, \Xi\vdash \Delta}
         \infer1{\Gamma, y, x, \Xi\vdash \Delta} }
       \qquad
       \tproof{ \hypo{\Gamma\vdash \Delta_1, R(\mathbf{x}), S(\mathbf{y}), \Delta_2}
         \infer1{\Gamma\vdash \Delta_1, S(\mathbf{y}), R(\mathbf{x}), \Delta_2} }
    \]

    Maintenant, les règles d'affaiblissement~:

    \[ \tproof{ \hypo{\Gamma\vdash\Delta}
         \infer1{\Gamma, x\vdash \Delta} }
       \qquad
       \tproof{ \hypo{\Gamma\vdash\Delta}
         \infer1{\Gamma\vdash\Delta,R(\mathbf{x})} }\]

    Enfin, les règles de contraction~:

    \[ \tproof{ \hypo{\Gamma,x,y\vdash\Delta}
         \infer1{\Gamma, x\vdash\Delta[y/x]} }
       \qquad
       \tproof{ \hypo{\Gamma\vdash\Delta, R(\mathbf{x}), R(\mathbf{x})}
         \infer1{\Gamma\vdash\Delta, R(\mathbf{x})} }\]

\end{defi}

On arrive maintenant au théorème de description syntaxique~:

\begin{theo}{Description syntaxique d'un morphisme de formule}
    Soient $F,G\in\C_\Sigma$ deux formules. Alors il existe une transformation
    naturelle $\theta:F\rightarrow G$ si et seulement si $<\phi_F>$ peut être réécrit
    en $<\phi_G>$ selon les règles de déduction ci-dessus.
\end{theo}

\begin{proof}
    TODO
\end{proof}

\begin{rem}
    On pourrait même tenter d'établir une bijection entre des arbres de preuve
    et des transformations naturelles, mais certains arbres donnent les même
    transformations, donc il faudrait quotienter par la bonne relation, ce qui
    n'est pas trivial.
\end{rem}


