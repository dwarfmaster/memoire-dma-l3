
\begin{defi}{Sous-canonicité}
    Un site $(C,J)$ est dit \emph{sous-canonique} sous tous les préfaisceaux
    représentables sont des faisceaux.
\end{defi}

On remarque qu'il est possible d'ordonner les sites sur une catégorie par inclusion
des recouvrements. On obtient alors un ordre partiel sur les sites, ce qui justifie
la définition suivante.

\begin{defi}{Canonicité}
    Un site sur une catégorie $C$ est dit canonique s'il est sous-canonique maximal.
\end{defi}

La proposition suivante justifie cette définition.

\begin{prop}
    Soit $C$ une catégorie. Il existe un unique site canonique sur cette catégorie.
\end{prop}

\begin{pv}
    Afin de prouver cette proposition, on va utiliser le lemme suivant~:

    \begin{lem}
        Une union quelconque de sites sous-canoniques est un site sous-canonique.
    \end{lem}
    \begin{pv}
        Commençons par remarquer qu'une union quelconque de sites est un site.
        La sous-canonicité vient du fait que la condition de faisceau est de la
        forme $\forall (f)\text{ recouvrement}, P(F,(f))$ où $F$ est le
        faisceau considéré.  Donc une union quelconque de familles de
        recouvrements qui ont toutes la propriétés vraie aura encore la
        propriété validée.
    \end{pv}

    L'unicité est alors immédiate.

    Pour l'existence, on peut construire le site canonique comme étant l'union
    de tous les sites sous-canoniques. Il suffit de vérifier qu'il existe toujours
    au moins un site sous-canonique. On prends le site discret~: chaque objet a
    exactement un recouvrement qui est son identité. C'est effectivement un site,
    et alors tout préfaisceau est un faisceau, donc notament les préfaisceaux
    représentables.
\end{pv}

Cette définition de canonicité rend compliqué le fait de vérifier qu'un site
spécifique la vérifie. Rien que vérifier la sous-canonicité peut s'avérer compliquer. On
va alors donner un critère de sous-canonicité avec lequel il est plus simple de
travailler.

\begin{defi}{Span sur $J$}
    Soit $J$ une famille de morphismes dans $\C$. Un span sur $J$ est un couple de
    morphismes $f:a\rightarrow b,g:a\rightarrow c$ de $\C$ de même codomaine tels que~:

    \[\exists f':b\rightarrow d, g':c\rightarrow d,\begin{tikzcd}
        & b \arrow[rd, "f'"] & \\
        a \arrow[ru, "f"] \arrow[rd, "g"] & & d \\
        & c \arrow[ru, "g'"] & \\
    \end{tikzcd}\]
\end{defi}

\begin{defi}{Famille effective épimorphique}
    Soit $J$ une famille de morphismes dans $\C$ de même domaine $c$. $J$ est dite
    \emph{effective polymorphique} si c'est la colimite du diagramme formé de tous
    les spans sur $J$.
\end{defi}

On a alors le résultat suivant~:

\begin{lem}
    Soit $(\C,J)$ un site. $(\C,J)$ est sous-canonique si et seulement si tous les
    récouvrements de $J$ sont effectifs-polymorphiques.
\end{lem}

\begin{pv}
    $\boxed{\implies}$ Soit $c\in \C$ et $S\in Jc$. Soit $D$ le diagramme formé
    de tous les spans sur $S$.

    Immédiatement, $(c,S)$ est un cocône sur $D$. Vérifions que c'est le cocône
    colimite.

    Soit $(c',S')$ un autre cocône sur $D$~:

    \[\begin{tikzcd}
        & d \arrow[rr, "u_d"] \arrow[rrdd, near start, "v_d"]
            & & c \arrow[dd, dashed, "?"] \\
        e \arrow[ru, "g"] \arrow[rd, "h"] & & & \\
        & d' \arrow[rruu, near start, "u_{d'}"] \arrow[rr, "v_{d'}"] & & c' \\
    \end{tikzcd}\]

    Remarquons que $C(\cdot, c')$ est un faisceau puisque le site est sous-canonique. On
    va donc construire la flèche manquante en utilisant la propriété de faisceau.
    On prend la famille $(v_d)_{d\in\text{dom}S}\in \Pi_{d\in\text{dom}S} C(d,c')$.
    
    C'est immédiatement une famille cohérente. En effet, la condition de cohérence
    reviens à dire que les $(v_d)$ commutent aux spans sur $S$, ce qui est exactement
    la façon dont ils sont définis.

    Alors la condition de faisceau nous donne l'existence d'une unique flèche dans
    $C(c,c')$ qui est un morphisme de cônes.

    $\boxed{\Longleftarrow}$ On peut en fait remonter la preuve puisque les familles 
    cohérentes sont exactement les cocônes sur les diagrammes de spans sur des
    recouvrements.
\end{pv}

\begin{cor}\label{caracCanon}
    On peut alors montrer que le site canonique est celui dont les recouvrements sont
    exactement les familles effectives épimorphiques de $\C$.
\end{cor}

