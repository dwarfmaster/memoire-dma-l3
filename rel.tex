
Cette partie contient une preuve non terminée du résultat suivant. Il est donc
possible que ce résultat soit faux. Heureusement il n'est utilisé nul part dans
ce mémoire, mais il serait probablement utile pour la construction de
$\gamma$ (voire \ref{secGamma}) si on avait eu le temps de la faire.

\begin{theo}{Relèvement sur $\tsigma$}\label{lifting}
    Soit $r : \C_\Sigma\rightarrow\tgamma$ un foncteur exact à droite. Alors il
    existe un foncteur exact à droite $\tilde{r}$ tel que~:

    \[\begin{tikzcd}
        \C_\Sigma\arrow[dr, bend right, "\iota"]\arrow[rr, bend left, "r"]
            & \cong & \tgamma \\
        & \tsigma\arrow[ur, bend right, "\tilde{r}"] & \\
    \end{tikzcd}\]
\end{theo}

\begin{pv}
    On commence par construire un $\tilde{r}$ candidat, puis on
    vérifiera qu'il est bien exact à droite.
    
    Soit $k\ast\xrightarrow{f} X\in\tsigma$. On note $r\ast = n\ast\rightarrow\phi$
    et $rX = m\ast\rightarrow\psi$. L'objet de $\tsigma$ étant une flèche de $\C_\Sigma$,
    elle est envoyée sur un carré commutatif. De plus, comme $r$ est exact à droite,
    on a $r(k\ast) = kr(\ast)$. Donc $r$ nous donne le carré commutatif suivant~:

    \[\begin{tikzcd}
        kn\ast\arrow[r, "r_1f"]\arrow[d]\arrow[dr,dashed] & m\ast\arrow[d] \\
        k\phi\arrow[r, "r_2f"] & \psi \\
    \end{tikzcd}\]

    On définit l'image de $f$ par $\tilde{r}$ comme étant la diagonale de ce carré
    commutatif. On doit maintenant définir l'image d'une flèche. Soit un carré commutatif
    dans $\tsigma$~:

    \[\begin{tikzcd}
        k_1\ast\arrow[r,"g_1"]\arrow[d,"f_1"] & k_2\ast\arrow[d,"f_2"] \\
        X_1\arrow[r,"g_2"] & X_2 \\
    \end{tikzcd}\]

    Comme l'image de chaque morphisme est un carré commutatif, donc en prenant
    l'image par $r$ on obtient la pyramide suivante~:

    \[\begin{tikzcd}
        k_1n\ast\arrow[rrr,red,"r_1g_1"]\arrow[dr]\arrow[ddd, "r_1f_1"]\arrow[ddr,dashed]
            & & & k_2n\ast\arrow[dl]\arrow[ddd, "r_1f_2"]\arrow[ddl,dashed] \\
        & k_1\phi\arrow[r,"r_2g_1"]\arrow[d,"r_2f_1"]
            & k_2\phi\arrow[d,swap,"r_2f_2"] & \\
        & \psi_1\arrow[r,red,"r_2g_2"]
            & \psi_2 & \\
        n_1\ast\arrow[ur]\arrow[rrr,"r_1g_2"]
            & & & n_2\ast\arrow[ul] \\
    \end{tikzcd}\]

    Les flèches en rouge nous donnent l'image du carré commutatif.

    On obtient facilement que $\tilde{r}$ préserve l'identité et la composition
    (on recolle les pyramides selon leur côté commun). C'est donc bien un foncteur.

    Construisons maintenant une transformation naturelle entre $rX$ et
    $\tilde{r}\iota X$. On définit $\theta_X$ comme étant le carré commutatif
    suivant~:

    \[\begin{tikzcd}
        kn\ast\arrow[r, "r_1f"]\arrow[d] & m\ast\arrow[d] \\
        \psi\arrow[r, "\id"] & \psi \\
    \end{tikzcd}\]

    Or $f$ est une flèche de $\tsigma$ créée par $\iota$, elle est donc bijective. Comme
    les foncteurs préservent les isomorphismes, $r_1f$ est un isomorphisme. On a donc
    une transformation inversible, vérifions qu'elle est naturelle~:

    \[\begin{tikzcd}
        kn\ast\arrow[rrrr, "r_1f"]\arrow[dr, "n\arf(h_\star)"]\arrow[dddddrr]
            & & & & m\ast\arrow[dl, swap, "r_1h"]\arrow[dddddll] \\
        & k'n\ast\arrow[rr, "r_1f'"]\arrow[ddr]
            & & m'\ast\arrow[ddl] & \\
        & & & & \\
        & & \psi' & & \\
        & & & & \\
        & & \phi\arrow[uu, "r_2h"] & & \\
    \end{tikzcd}\]

    La base et le sommet commutent. De plus, le côté droit et celui du haut commutent.
    Donc donc les faces commutent~: $\theta$ est un isomorphisme naturel entre
    $\tilde{r}\circ\iota$ et $r$.

    Il faudrait vérifier que $\tilde{r}$ est exact à droite pour conclure.
\end{pv}


