
Soit $\Sigma = \{R_1, \dots R_n\}$ un alphabet de symboles de relations, avec
leurs arités $a_1, \dots a_n\in\N$.

L'objectif est de créer une catégorie $\C_\Sigma$, que l'on notera $\C$ en l'absence
d'ambiguïté sur $\Sigma$, telle que la donnée d'un problème CSP soit exactement la donnée
d'un faisseau sur $\C$ (à valeur dans $\fset$, puisqu'on se limite à ce cadre là). Pour
pouvoir parler de faisseau, il faudra munir $\C$ d'une structure de site.

\subsection{Définition de $\C$}

Afin de simplifier, on va essayer de donner une représentation des formules indépendante
du renommage des variables (pour éviter les problèmes d'alpha-équivalence) et qui réduit
la structure d'une formule à son plus simple.

La donnée d'une formule est alors un $n$-uplet $r\in\N^n$, avec $r_1$ qui indique le
nombre d'occurence du prédicat $R_1$, $r_2$ le nombre d'occurences du prédicat $R_2$,
\dots On ne dit pour l'instant rien sur les variables, et on appelle $r$ le 
\emph{squelette} de la formule.

\begin{defi}{Lieu de variable}
    Étant donné $r\in\N^n$ un squelette de formule, on va définir l'ensemble des
    \emph{lieux de variables} de la façon suivante~:

    \[ \V(r):= \{(i,j,k)\in\N^3 |        1\leq i\leq n
                                  \wedge 1\leq j\leq r_j
                                  \wedge 1\leq k\leq a_j\} \]

    Intuitivement, un triplet $(i,j,k)\in\V(r)$ désigne un endroit où une variable peut
    aller dans la formule, plus précisément cela désigne le $k$-ième argument de la
    $j$-ième instance d'un prédicat $R_i$.
\end{defi}

Dans une formule, tous les emplacement possibles ne sont pas nécessairement remplis
par des variables différentes, il faut donc un moyen d'assimiler les lieus de variable
qui sont occupés par la même variable. On fait ça de façon brutale en se munissant d'une
relation d'équivalence sur $\V(r)$.

\begin{defi}{Formule}
    Une \emph{formule} est alors un couple $(r,R)\in\N^n\times\P(\V(r)^2)$ où $R$ est
    une relation d'équivalence. On nomme une \emph{variable} de $(r,R)$ une classe
    d'équivalence de lieus, c'est à dire un élément de $\V(r)/R$.
\end{defi}

La catégorie $\C$ va alors avoir comme objets les formules telles que définit ci-dessus,
plus un objet spécial $\star$.

Il faut maintenant définir une notion de morphisme sur $\C$. On va distinguer les
morphismes entre formule, et les morphismes depuis $\star$, qui a un rôle particulier.
Commençons par les morphismes depuis $\star$. Ces morphismes sont simples~: ils 
représentent les variables de la formules visée.

\begin{defi}{Morphismes depuis $\star$}
    On pose $\C(\star,\star) = \{\text{id}_\star\}$ (cet objet n'a pas d'intérêt autre
    que d'être nécessaire). Plus intéressant, on pose~:

    \[\C(\star, (r,R)) = \V(r)/R \]
\end{defi}

Maintenant, il faut définir les morphismes entre formules. L'idée est qu'un morphisme
d'une formule vers une autre est une instance de la première formule dans la deuxième.
Intuitivement, ça indique que la première formule est une sous-formule de la deuxième,
et de quelle manière. Pour cela, il faut un moyen d'indiquer où vont les anciennes
instances des relations. On fait ça à l'aide d'un \emph{déplacement}.

\begin{defi}{Déplacement}
    Un \emph{déplacement} de $r$ vers $r'$ est un n-uplet de fonctions $\phi$, tel que
    $\forall 1\leq i\leq n, \phi_i : r_i\hookrightarrow r_i'$.

    On note l'ensemble des déplacements de $r$ vers $r'$ $\D(r,r')$
\end{defi}

\begin{rem}
    On remarque que $\D(r,r')\neq\emptyset\implies r\leq r'$ où $\leq$ est l'ordre
    partiel point par point.
\end{rem}

On va maintenant définir l'action d'un déplacement sur une relation.

\begin{defi}{Déplacement d'une relation}
    Soient $(r,R)$ et $(r',R')$ deux formules. Soit $\phi\in\D(r,r')$. On pose~:

    \[ \begin{array}{rccl}
          \phi(R) &:= & & \{ ((i,\phi_i(j), k), (i', \phi_{i'}(j'), k'))
                           ~|~ ((i,j,k), (i',j',k'))\in R \} \\
                  &   & \cup &\{ ((i,j,k), (i,j,k)) ~|~ (i,j,k)\in\V(r') \} \\
    \end{array} \]
\end{defi}

On peut vérifier que cette définition a bien un sens~:

\begin{lem}
    Soient $(r,R)$ et $(r',R')$ deux formules. Soit $\phi\in\D(r,r')$.
    Alors $\phi(R)$ est une relation d'équivalence sur $\V(r')$.
\end{lem}

\begin{proof}
    TODO mais facile.
\end{proof}

On a maintenant tout en main pour définir les morphismes entre formules.

\begin{defi}{Morphismes entre formules}
    Soient $(r,R)$ et $(r',R')$ deux formules. On pose~:

    \[\C((r,R),(r',R')) := \{ \phi\in\D(r,r') ~|~ \phi(R)\subseteq R' \}\]
\end{defi}

\begin{rem}\begin{itemize}
    \item $\C((r,R), (r',R'))\neq\emptyset \implies r\leq r'$, ce qui est cohérent
        avec l'intuition derrière la définition des morphismes.
    \item $\C((r,R), (r,R))$ n'est jamais vide, il contient toujours le $n$-uplet
        des identités des $r_i$. Ce sera d'ailleurs la flèche identité pour la loi
        de composition que l'on définira plus bas.
    \item $\star$ n'a aucune flèche entrante en dehors de son identité.
\end{itemize}\end{rem}

On a maintenant besoin de quelques définitions et lemmes préliminaires pour pouvoir
définir proprement la composition de morphismes dans $\C$.

\begin{lem}
    Soient $(r,R)$, $(r',R')$ et $(r'',R'')$ des formules.
    Soient $\phi_1\in\C((r,R),(r',R'))$ et $\phi_2\in\C((r',R'),(r'',R''))$.
    Alors $\phi_2\circ\phi_1\in\C((r,R),(r'',R''))$ où $\circ$ est compris
    terme à terme.
\end{lem}

\begin{proof}
    Comme la composition de deux injections est un injection, on a que
    $\phi_2\circ\phi_1\in\D(r,r'')$.

    On sait que l'on a $\phi_1(R)\subseteq R'$ et $\phi_2(R')\subseteq R''$. Or
    les déplacements sont immédiatements croissants pour l'inclusion lorsque pensés
    comme fonctions de relations.
    Donc $\phi_2\circ\phi_1(R)\subseteq\phi_2(R')\subseteq R''$.
\end{proof}

\begin{defi}{Déplacement d'un lieu de variable}
    Soient $(r,R)$ et $(r',R')$ des formules, et $\phi\in\D(r,r')$.
    Soit $(i,j,k)\in\V(r)$. On définit $\phi((i,j,k)) = (i, \phi_i(j), k)\in\V(r')$.
\end{defi}

\begin{lem}
    Soient $(r,R)$ et $(r',R')$ des formules, et $\phi\in\C((r,R),(r',R'))$.
    On a que~:
    
    \[\forall v_1,v_2\in\V(r), (v_1,v_2)\in\R\implies (\phi(v_1),\phi(v_2))\in R'\]
\end{lem}

\begin{defi}{Déplacement d'une variable}
    Soient $(r,R)$ et $(r',R')$ des formules, et $\phi\in\C((r,R),(r',R'))$.
    Soit $c\in\V(r)/R$. Soit $v\in c$. On pose $\phi(c) = [\phi(v)]\in\V(r')/R'$.

    Le lemme précédent nous assure que c'est bien définit.
\end{defi}

On a maintenant tout en main pour définir la composition de morphismes.

\begin{defi}{Composition dans $\C$}
    Il y a trois cas à considérer~:\begin{itemize}
        \item $id_\star\in\C(\star,\star)$ et $f\in\C(\star, x)$ où $x$ est n'importe
            quel object de $\C$. Alors $f\circ id_\star := f$.
        \item $c\in\C(\star, (r,R))$ et $\phi\in\C((r,R), (r',R'))$. Alors
            on pose $\phi\circ c := \phi(c)$.
        \item $\phi_1\in\C((r,R),(r',R'))$ et $\phi_2\in\C((r',R'),(r'',R''))$. Alors
            on pose $\phi_2\circ\phi_1$ définit terme à terme, comme dans le lemme
            précédent.
    \end{itemize}
\end{defi}

\begin{lem}
    Pour toute formule $(r,R)$, on a que
        $\phi=(id_{r_1},\dots id_{r_n})\in\C((r,R),(r,R))$.
    De plus, $\forall f\in\C(x, (r,R)), \phi\circ f = f$
    et $\forall \phi'\in\C((r,R), (r',R')), f\circ\phi = f$.
\end{lem}

\begin{lem}
    Soient $(r,R)$, $(r',R')$ et $(r'',R'')$ trois formules. Soit $c\in\C(\star,(r,R))$,
    $\phi_1\in\C((r,R), (r',R'))$ et $\phi_2\in\C((r',R'), (r'',R''))$.

    Alors $(\phi_2\circ\phi_1)\circ c = \phi_2\circ (\phi_1\circ c)$.
\end{lem}

\begin{lem}
    Soient $(r_1,R_1)$, $(r_2,R_2)$, $(r_3,R_3)$ et $(r_4,R_4)$ quatres formules.
    Soient $\phi_1\in\C((r_1,R_1), (r_2,R_2))$, $\phi_2\in\C((r_2,R_2), (r_3,R_3))$ et
    $\phi_3\in\C((r_3,R_3),(r_4,R_4))$.

    Alors $\phi_3\circ(\phi_2\circ\phi_1) = (\phi_3\circ\phi_2)\circ\phi_1$.
\end{lem}

\begin{proof} Les trois lemmes sont immédiats. Il suffit d'écrire la définition de $\circ$
    dans chacun des cas pour que ça marche.
\end{proof}

\begin{theo}{Définition de $\C$}
    $\C$ est une catégorie.
\end{theo}

\begin{proof} Conséquence immédiate des trois lemmes précédents.
\end{proof}
\subsection{Structure de site}

\begin{defi}{Site}
    Soit $\C$ une catégorie quelconque. On donne, pour
    chaque $c\in\C$, d'un ensemble de familles $(f_i : c_i\rightarrow c)_{i\in I}$,
    dites couvrantes, telles que~:

    \begin{align*}
        \forall c\in\C, & \forall (f_i)_{i\in I} \text{ couvrante de } c,
          \forall h : d\rightarrow c, \exists (g_j)_{j\in J} \text{ couvrante de } d, \\
        & \forall j\in J, \exists i\in I,
          \begin{tikzcd}[ampersand replacement=\&]
            d \arrow[r, "h"] \& c \\
            g_j \arrow[u, "g_j"] \arrow[r, dotted] \& c_i \arrow[u, "f_i"] \\
          \end{tikzcd}
    \end{align*}

    La catégorie $\C$ munie de ces familles est appellée un \emph{site}.
\end{defi}

L'intuition est de définir une notion de recouvrement d'objets de $\C$ par des morphismes.
Cela permet de définir une notion de localité sur les objets de $\C$, sans pour autant
avoir de foncteur d'oublie vers $\Set$ et de topologie. Cela permet de définir des
faisceaux, qui sont intuitivement une manière d'associer une collection d'objets à
chaque objet de $\C$ qui soit essentiellement locale.

\begin{defi}{Famille cohérente}
    Soit $\C$ un site et $F : \C^{\text{op}}\rightarrow\Set$ un préfaisceau sur
    $\C$. Soit $c\in\C$ et $(f_i)_{i\in I}$ une famille couvrante de $c$.
    
    Soit $(a_i)_{i\in I}\in \Pi_{i\in I} Fc_i$. $(a_i)$ est dite \emph{cohérente} si~:

    \[\forall i,j\in I, \forall f : d\rightarrow c_i, \forall h : d\rightarrow c_j,
      \begin{tikzcd}
          d \arrow[r, "g"] \arrow[d, "h"] & c_i \arrow[d, "f_i"] \\
          c_j \arrow[r, "f_j"] & c \\
      \end{tikzcd}
      \implies Fg(a_i) = Fh(a_j) \]
\end{defi}

Si l'on pense une famille couvrante comme une notion de localité, une famille cohérente
est une information locale sur un objet $Fc$.

\begin{defi}{Faisceau}
    Un faisceau sur $\C$ un site est un préfaisceau sur $\C$ tel que~:

    \begin{align*}
        \forall c\in\C, & \forall (f_i)_{i\in I}\text{ couvrante de } c, \\
                        & \forall (a_i)_{i\in I}\in\Pi_{i\in I}Fc_i \text{ cohérente}, \\
                        & \exists ! a\in Fc, \forall i\in I, Ff_i(a) = a_i
    \end{align*}
\end{defi}

La condition de faisceau confirme l'intuition~: elle nous dit que l'information locale
correspont bien à un objet de $Fc$. Sans cette condition on ne pourrait pas nécessairement
obtenir l'existence d'un objet à partir d'une description locale.

Pourquoi s'intéresser à ça ? Parce que l'objectif est de montrer que les CSP sur un 
alphabet de relation $\Sigma$ sont les faisceau sur la catégorie $\C_\Sigma$,
où $F\phi$ contient des témoins de satisfiabilité de $\phi$. La construction de $\C_\Sigma$
et la condition de fonctorialité nous permettrons de dire que chaque élément de $F\phi$
correspond bien à une solution de $\phi$, mais ce n'est pas suffisant pour obtenir que
$F\phi$ contient des témoins pour toutes les solutions. Pour cela, il faut montrer que
l'on peut construire une solution de $\phi$ à partir de ses constituants~: c'est ce que
la condition de faisceau va nous donner.

On veut donc définir une notion de recouvrement sur $\C_\Sigma$. L'idée est qu'un
recouvrement d'une formule doit être suffisant pour obtenir toutes les solutions de
cette formule. Il faut donc avoir les solutions de chaque formule atomique, directement
ou via une formule plus compliqué. On va maintenant formaliser ça, la preuve que ça
fonctionne sera donnée dans la section suivante.

\begin{defi}{Recouvrement d'une formule}
    Soit $(r,R)$ une formule dans $\C_\Sigma$. Soit un $(\phi_i)_{i\in I}$ un ensemble
    de morphismes de formules de $\C_\Sigma$ (ne contient donc pas de flèche depuis
    $\star$). $(\phi_i)$ est dit \emph{couvrante} si~:

    \[ \bigcup_{i\in I} \im\phi_i = r \]

    Où l'image de $\phi$ est le $n$-uplet des images des fonctions constituantes et
    l'union est prise terme à terme.
\end{defi}

Cette définition revient exactement à dire que chaque formule atomique de la formule est
présente dans au moins une des formule du recouvrement.

\begin{defi}{Site sur $\C_\sigma$}
    On munit $\C_\Sigma$ d'une structure de site en lui donnant, pour chaque formule,
    tous les recouvrements définis tels que précédemment, et pour $\star$ le recouvrement
    $(\id_\star)$.
\end{defi}

Il faut prouver que cette définition donne bien un site. C'est l'objectif du
lemme suivant.

\begin{rem}
    Soit $(r,R)$ une formule de $\C_\Sigma$. Alors cette formule possède un
    recouvrement constitué uniquement de formule atomique avec une relation d'équivalence
    triviale. On appelle ce recouvrement le \emph{recouvrement fondamental}
    (il est unique à permutation de la famille près).
\end{rem}

\begin{lem}
    Cette définition est bien un site.
\end{lem}

\begin{proof}
    Soit $\star\in\C$. Comme il existe une seule flèche entrante en $\star$, son identité,
    et un seul recouvrement, la propriété de site est immédiatement vraie.

    Soit $(r,R)$ une formule et $(f_i)_{i\in I}$ un recouvrement de cette formule.
    
    Soit $\phi$ un morphisme de $(r', R')$ dans $(r,R)$. On prend $(g_j)_{j\in J}$ le
    recouvrement fondamental de $(r', R')$. Soit $j\in J$. On a que $\phi\circ g_j$ est
    l'une des sous-formules atomiques de $(r,R)$. Par définition d'un recouvrement, on
    a $i\in I$ tel que $f_j$ identifie une sous-formule de $(r,R)$ contenant la
    sous-formule atomique considérée. Le morphisme qui envoie cette formule atomique
    dans le départ de $f_j$ fait alors commuter le carré voulu.

    Soit $v$ un morphisme de $\star$ dans $(r,R)$, autrement dit une variable de $(r,R)$.
    Cette variable apparait nécessairement dans une sous-formule du recouvrement. En
    prenant le morphisme depuis $\star$ vers cette sous-formule qui identifie l'une des
    variables qui est ensuite envoyée sur $v$, on fait commuter le carré voulu.
\end{proof}

\subsection{Correction de la définition}

On veut maintenant

\subsubsection{Tout CSP est un faisceau}

\begin{defi}{CSP}
    Un CSP est une sémantique finie pour la logique régulière sur un alphabet de formules 
    $\Sigma = \{R_1, \dots R_n\}$. Concrètement, cela consiste en un domaine $D$ qui est
    un ensemble fini, et pour chaque $1\leq i\leq n$ une relation $\sem{R_i}$
    d'arité $a_i$ sur $D$.
\end{defi}

\begin{rem}
    On considère une formule conjonctive $\psi$. On note alors $\V(\psi)$ ses variables.
    Pour chaque formule atomique $R_i(x_1, \dots x_{a_i})$ de $\psi$, on dispose
    d'une fonction de $\V(R_i(x_1, \dots x_{a_i}))\rightarrow\V(\psi)$
    qui nous donne l'assimilation des variables de $R_i$ vue comme sous formule de
    $\psi$ (et ce pour chaque instance de $R_i$ dans $\psi$).

    Alors $\sem{R_i(x_1, \dots x_{a_i})}$ peut être vu comme une partie de
    $\fns(\V(R_i(\dots)), D)$.
\end{rem}

\begin{defi}{Extension de domaine}
    Étant donné deux ensembles de variables $V_1$ et $V_2$ avec une fonction
    $i : V_1 \rightarrow V_2$ et une partie $E\subseteq\fns(V_1, D)$. On peut définir
    l'extension de $E$ par rapport à $i$ de la façon suivante~:

    \[ \ext_i(E) = \{f\in\fns(V_2,D) | \exists g\in E, f_{|i(V_1)} = g\circ i \} \]
\end{defi}

Intuitivement, cela correspond à rajouter des variables qui n'apparaissent pas dans une
formule à sa sémantique. On dispose alors d'un moyen de définir la sémantique d'une
formule à partir de ses formules atomiques.

\begin{defi}{Sémantique d'une formule}
    Soit $\psi$ une formule quelconque. $\psi$ s'écrit $A_1\wedge \dots \wedge A_k$ où
    les $A_i$ sont des formules atomiques. On note $j_i$ l'entier tel que $R_{j_i}$ soit
    la relation de $A_i$. On note $f_i$ la fonction de $\V(R_{j_i}(x_1, x_2 \dots))$
    dans $\V(\psi)$ telle que décrite dans la remarque ci-dessus. On pose alors~:

    \[ \sem{\psi} = \bigcap_{1\leq i\leq k} \ext_{f_i}(\sem{R_{j_i}(x_1, x_2\dots)}) \]

    Il est facile de se convaincre que cette sémantique correspond bien à la sémantique
    telle que donnée par la structure de logique régulière.
\end{defi}

On peut alors définir un faisceau sur $\C_\Sigma$ à partir du CSP. On remarque que pour 
chaque formule $(r,R)$ de $\C_\Sigma$, on peut associer une vraie formule conjonctive
$\psi_{(r,R)}$ de manière naturelle-ish. %TODO détailler l'association

\begin{defi}{Faisceau de CSP}
    % TODO sans intérêt sans l'association précédente
    On pose $F\star = D$ et $F(r,R) = \sem{\psi_{(r,R)}}$.

    Soit $v\in\C_\Sigma(\star, (r,R))$, on peut lui associer une variable $v'$ de manière
    naturelle dans $\V(\psi_{(r,R)})$. Alors~:
    
    \[ Fv = \left\{\begin{array}{ccc}
                       \sem{\psi_{(r,R)}} & \rightarrow & D     \\
                              f           & \mapsto     & f(v') \\
    \end{array}\right.\]
\end{defi}

\subsubsection{Tout faisceau est un CSP}


