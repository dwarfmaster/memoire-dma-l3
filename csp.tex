
Soit $D$ un ensemble fini.

\subsection{CSP}

\begin{defi}{Logique régulière}
    Une proposition de la logique régulière est construite à partir d'un nombre
    fini de relations $R_1, \dots R_n$, sur un langage de termes ne contenant que
    des variables $x, y, z \dots$, de la conjonction, de la quantification existentielle
    et du symbole faux.

    \[\begin{array}{rcl}
        P, Q & := & R_1(x_1, \dots, x_{\ar(R_1)}) \\
             & |  & \vdots                        \\
             & |  & R_n(x_1, \dots, x_{\ar(R_n)}) \\
             & |  & x = y                         \\
             & |  & P \wedge Q                    \\
             & |  & \exists x. P                  \\
             & |  & \bot                          \\
    \end{array}\]
\end{defi}

\begin{defi}{Problème de satisfaction de contraintes}
    Pour un ensemble fini $\mathcal{D}$ de contraintes, on donne une interprétation de
    ces contraintes sur $D$. Cela revient à dire que $D$ est un modèle de la théorie
    vide sur le langage de la logique régulière. L'interprétation des différents 
    connecteurs se font comme en théorie des modèle de premier ordre usuel. 
    $\csp(\mathcal{D})$ est alors le problème, pour une proposition de la logique
    régulière $P$, de déterminer si $D\models P$.
\end{defi}

Par la suite, on assimilera une proposition à $m$ variables libres avec une relation
d'arité $m$.

\begin{lem}
    Pour tout $\mathcal{D}$, $\csp(\mathcal{D})$ est dans $NP$.

    On ne peut pas dire mieux dans le cas général. En effet, $3-SAT$ est un $\csp$
    (facile), donc certains $\csp$ sont $NP$-complets.
\end{lem}

\begin{theo}{Dichotomie}
    Soit $\mathcal{D}$. $\csp(\mathcal{D})$ est soit $NP$-complet, soit dans $P$.
\end{theo}

Ce théorème est très difficile. Pour commencer à l'attauqer, on veut pouvoir comparer les
$\csp$ par difficulté. La comparaison qui va nous intéresser est la réduction LOGSpace.

\begin{defi}{Réduction}
    Soient $\mathcal{D}$ et $\mathcal{F}$ des ensembles finis de relations, éventuellement
    sur des domaines différents. On écrit $\csp(\mathcal{D})\leq\csp(\mathcal{F})$ si il
    existe une fonction calculable en espace logarithmique qui envoie une instance de
    $\csp(\mathcal{D})$ sur une instance de $\csp(\mathcal{F})$ qui soit équisatisfiable.
\end{defi}

On va maintenant considérer des critères plus simples qui permettent de comparer des
$\csp$.

\begin{defi}{pp-réduction}
    $\mathcal{F}$ est dit pp-réductible (primitif positif réductible) à
    $\mathcal{D}$ (sur un même domaine) si chaque relation de $\mathcal{F}$ peut
    s'écrire comme une formule régulière sur $\mathcal{D}$.
\end{defi}

\begin{lem}
    Si $\mathcal{F}$ est pp-réductible à $\mathcal{D}$, alors
    $\csp(\mathcal{F})\leq\csp(\mathcal{D})$.
\end{lem}

\subsection{Adjunction}

\begin{defi}{$\Rel$}
    $\Rel$ est l'ensemble des ensembles de relations sur $D$.

    \[\Rel = \P\left(\bigcup_{n\in\N^*} \P(D^n)\right)\]
\end{defi}

\begin{defi}{$\Oper$}
    $\Oper$ est l'ensemble des ensemble d'opérations sur $D$.

    \[\Oper = \P\left(\bigcup_{n\in\N^*} \P(D^{D^n})\right)\]
\end{defi}

Ces deux ensembles sont partiellement ordonnables par l'inclusion. On va alors
chercher à créer une connection de Galois (ou adjonction/dualité) entre ces
ensembles.

Pour cela, on va définir une relation entre les opérations et les relations sur $D$.

\begin{defi}{$\bot$} Soit $R\in \bigcup_{n\in\N^*} \P(D^n)$ une relation sur $D$
    et $f\in\bigcup_{n\in\N^*} \P(D^{D^n})$ une opération sur $D$. On écrit alors
    $R\bot f$ si, en posant $n = \ar(R)$ et $m = \ar(f)$~:

    \[ \forall (a_{i,j})\in D^{nm}, (\forall 1\leq j\leq m, R(a_{1,j}, \dots a_{n, j}))
         \implies R(f(a_{1,1}, \dots a_{1,m}), \dots f(a_{n,1}, \dots a_{n,m})) \]
\end{defi}

On va maintenant définir les deux foncteurs contrinvariant (fonctions décroissantes)
entre $\Rel$ et $\Oper$ qui formeront notre adjunction.

\begin{defi}{$\Pol$ et $\Inv$}
    On définit~:

    \[\Pol:\begin{array}{rcl}\Rel & \rightarrow & \Oper \\
        S & \mapsto & \{f\in\Oper | \forall R\in S, R\bot f\} \\
    \end{array}\]

    \[\Inv:\begin{array}{rcl}\Oper & \rightarrow & \Rel \\
        S & \mapsto & \{R\in\Rel | \forall f\in S, R\bot f\} \\
    \end{array}\]
\end{defi}

\begin{lem}
    $\Pol$ et $\Inv$ sont des foncteurs contrinvariants, respectivement de $\Rel$ vers
    $\Oper$ et de $\Oper$ vers $\Rel$.
\end{lem}

On a maintenant le premier résultat intéressant, qui motive leur définition~:

\begin{theo}{Adjunction}
    $\Pol\dashv\Inv : \Rel^\text{op} \rightarrow \Oper$
\end{theo}
\begin{proof}
    Soit $S_R\in\Rel$ et $S_f\in\Oper$. On a~:

    \begin{align*}
        S_f\subseteq\Pol(S_R) &\iff \forall f\in S_f, \forall R\in S_R, R\bot f \\
                              &\iff \forall R\in S_R, \forall f\in S_f, R\bot f \\
                              &\iff S_R\subseteq\Inv(S_f)
    \end{align*}
\end{proof}

\subsection{Clone}

\begin{defi}{Clone} Un clone est un ensemble d'opérations (donc un élément de $\Oper$)
    contenant l'identité, et stable par composition, permutation, affaiblissement et
    contraction.
\end{defi}

\begin{defi}{Clone relationel} Un clone relationel est un ensemble de relations
    (donc un élément de $\Rel$) contenant l'égalité et clôt par les constructions
    régulières (quantification existentielle et conjonction).
\end{defi}

\begin{defi}{Clone relationel engendré} Pour $\Gamma\in\Rel$, on note $<\Gamma>$ le plus
    petit clone relationel contenant $\Gamma$ (c'est bien défini car l'intersection d'un
    ensemble quelconque de clones relationels est un clone relationel).
\end{defi}

\subsection{Théorème important}

% TODO hypothèse : l'image de Inv est toujours un clone relationel, et l'image de Pol est
% toujours un clone
\begin{lem}{}
    \[\forall \Gamma\in\Rel, \Inv(\Pol(\Gamma)) \text{ est un clone relationel}\]
\end{lem}
\begin{proof}
    Soit $\Gamma\in\Rel$.

    Soit $f$ une opération sur $D$. Comme
    $\forall (a_i), (b_i)\in D^{\ar(f)}, (a_i) = (b_i) 
        \implies f(a_1, \dots a_{\ar(f)}) = f(b_1, \dots b_{\ar(f)})$, on a $\Delta\bot f$
    et donc $\Delta\in \Inv(F)$ pour tout $F\in\Oper$. Donc $\Delta\in\Inv(\Pol(\Gamma))$.

    De manière immédiate, $\Inv(F)$ est clos par intersection de relations pour tout
    $F\in\Oper$.

    Soit $f\in\Pol(\Gamma)$. Soit $R\in\Inv(\Pol(\Gamma))$, avec $\ar(R)>1$. Posons
    $R'$ la relation définie par $R'(x_1, \dots x_{\ar(R)-1}) :\iff
    \exists x. R(x, x_1, \dots x_{\ar(R)-1})$. Posons $m = \ar(R) - 1$ et $n = \ar(f)$.
    Soient maintenant $(a_{i,j})\in D^{mn}$ tels que
    $\forall 1\leq j\leq n, R'(a_{1,j}, \dots, a_{m,j})$. On prends $x_1, \dots x_n$ les
    témoins de chaque relation. En posant $x = f(x_1, \dots x_n)$, et
    $a_i = f(a_{i,1}, \dots a_{i,n})$, on a $R(x, a_1, \dots a_m)$ et donc
    $R'(a_1, \dots a_m)$, d'où le résultat.
\end{proof}

\begin{theo}{1969}
    \[\forall \Gamma\in\Rel, <\Gamma> = \Inv(\Pol(\Gamma))\]
\end{theo}
\begin{proof}\begin{description}
    \item[$\subseteq$] Par propriété d'adjonction, on a 
        $\Gamma\subseteq\Inv(\Pol(\Gamma))$. Or d'après le lemme précédent, 
        $\Inv(\Pol(\Gamma))$ est un clone relationel. Donc, comme $<\Gamma >$ est le
        plus petit clone relationel contenant $\Gamma$, on a 
        $<\Gamma>\subseteq\Inv(\Pol(\Gamma))$.
    \item[$\supseteq$] Difficile ! Donc admis. % TODO
\end{description}\end{proof}

\begin{cor}{Critère de pp-réductibilité}
    $\mathcal{D}$ est pp-réductible à $\mathcal{F}$ si et seulement si
    $\Pol(\mathcal{F})\subseteq\Pol(\mathcal{D})$.
\end{cor}

\begin{proof}
    $\mathcal{D}$ est pp-réductible à $\mathcal{F}$ si et seulement si
    $\mathcal{D}\subseteq<\mathcal{F}>$ (c'est exactement la définition de
    pp-réductibilité). On a alors~:
    \[\begin{array}{rll}
        \mathcal{D}\subseteq<\mathcal{F}>
          &\iff \mathcal{D}\subseteq\Inv(\Pol(\mathcal{F}))
            & \text{ par le théorème précédent} \\
          &\iff \Pol(\mathcal{F})\subseteq\Pol(\mathcal{D})
            & \text{ par propriété d'adjunction} \\
    \end{array}\]
\end{proof}



