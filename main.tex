\documentclass[12pt]{article}

\input{inc.tex}

\usepackage{mathrsfs}
\usetikzlibrary{positioning}
\usepackage{tikz-cd}
\usepackage{ebproof}
\newcommand\M{\text{M}(X\uplus X^{-1})}
\newcommand\G{\mathscr{G}(X)}
\renewcommand\F{\mathscr{F}(X)}
\newcommand\V{\mathbb{V}}
\newcommand\E{\mathbb{E}}
\newcommand\D{\mathbb{D}}
\newcommand\bet{\rightarrow_\beta}
\newcommand\beq{=_\beta}
\newcommand\Rel{\text{Rel}}
\newcommand\Set{\text{Set}}
\newcommand\Oper{\text{Oper}}
\newcommand\Inv{\text{Inv}}
\newcommand\Pol{\text{Pol}}
\renewcommand\P{\mathscr{P}}
\newcommand\ar{\text{ar}}
\newcommand\arf{\text{ar}}
\newcommand\csp{\text{CSP}}
\renewcommand\C{\mathscr{C}}
\newcommand\fset{\text{FinSet}}
\newcommand\ffun{\text{FinFun}}
\newcommand\im{\text{Im }}
\newcommand\sem[1]{\llbracket {#1} \rrbracket}
\newcommand\ext{\text{Ext}}
\newcommand\fns{\mathscr{F}}
\newcommand\brk[1]{[ {#1} ]}
\newcommand\psh{\text{PSh}}
\newcommand\sh{\text{Sh}}
\newcommand\shc{\text{Sh}_\text{can}}
\newcommand{\tproof}[1]{{\scantokens{\begin{prooftree}#1\end{prooftree}}}}
\newcommand{\cf}{\mathbb{F}}
\newcommand{\tsigma}{\widetilde{\Sigma}}
\newcommand{\tgamma}{\widetilde{\Gamma}}
\newcommand{\tlambda}{\widetilde{\Lambda}}
\newcommand{\colim}{\text{colim}}
\newcommand{\fcsp}{\text{FCSP}}
\newcommand{\op}{\text{op}}
\newcommand{\tf}{\widetilde{F}}
\newcommand{\tg}{\widetilde{G}}
\newcommand{\tH}{\widetilde{H}}

\setcounter{tocdepth}{3}

\title{Mémoire de L3~: Un approche catégorique au problème CSP}
\author{Luc Chabassier\and Simon Cabuche}
\date{\small Sous la direction de Damiano Mazza}

\begin{document}

\maketitle

\newpage
\tableofcontents
\newpage


%   ____ ____  ____                    _     _                
%  / ___/ ___||  _ \   _ __  _ __ ___ | |__ | | ___ _ __ ___  
% | |   \___ \| |_) | | '_ \| '__/ _ \| '_ \| |/ _ \ '_ ` _ \ 
% | |___ ___) |  __/  | |_) | | | (_) | |_) | |  __/ | | | | |
%  \____|____/|_|     | .__/|_|  \___/|_.__/|_|\___|_| |_| |_|
%                     |_|                                     
\section{Problèmes de satisfaction de contraintes}\label{secCSP}
\input csp.tex

%   ____      _                        _           _    ____ ____  ____  
%  / ___|__ _| |_ ___  __ _  ___  _ __(_) ___ __ _| |  / ___/ ___||  _ \ 
% | |   / _` | __/ _ \/ _` |/ _ \| '__| |/ __/ _` | | | |   \___ \| |_) |
% | |__| (_| | ||  __/ (_| | (_) | |  | | (_| (_| | | | |___ ___) |  __/ 
%  \____\__,_|\__\___|\__, |\___/|_|  |_|\___\__,_|_|  \____|____/|_|    
%                     |___/                                              
\section{Une catégorie pour les CSP}\label{secCat}
\input sheaves.tex

\section{PP-interprétabilité catégorique}\label{secPP}
\input pp.tex

\section{Conclusion}

Suite à ces travaux, on a presque fini de reformuler en terme de théorie des
catégories les notions et résultats principaux liés à l'étude des CSPs. Une
fois cette reformulation terminée (et surtout le lien entre notre formulation
et la théorie standard mieux compris), il serait intéressant de continuer en
transposant la preuve du théorème de dichotomie des CSPs dans notre
formulation. Notre espoir est que une fois cette reformulation finie, on pourra
expliciter précisément les propriétés nécessaires des objets construits, pour
les axiomatiser et voir si on ne peut pas étendre le résultat à une classe plus
large de problème.

\section*{Remerciements}

Nous tenons à remercier Damiano Mazza pour nous avoir encadré tout au long du
mémoire et nous avoir permi d'avancer autant.


%     _                               _ _               
%    / \   _ __  _ __   ___ _ __   __| (_) ___ ___  ___ 
%   / _ \ | '_ \| '_ \ / _ \ '_ \ / _` | |/ __/ _ \/ __|
%  / ___ \| |_) | |_) |  __/ | | | (_| | | (_|  __/\__ \
% /_/   \_\ .__/| .__/ \___|_| |_|\__,_|_|\___\___||___/
%         |_|   |_|                                     
\appendix
\newpage
\section{Canonicité}\label{appCanon}
\input canon.tex

\newpage
\section{Approche naïve de catégorie pour les CSP}\label{appNaif}
\input naif.tex

\newpage
\section{Descriptions syntaxiques}\label{appSyn}
\input syntaxic.tex

\newpage
\section{Vision algébrique}\label{appClones}
\input clones.tex

\end{document}

