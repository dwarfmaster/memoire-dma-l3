\documentclass[12pt]{article}

\input{inc.tex}

\usepackage{mathrsfs}
\usetikzlibrary{positioning}
\newcommand\M{\text{M}(X\uplus X^{-1})}
\newcommand\G{\mathscr{G}(X)}
\renewcommand\F{\mathscr{F}(X)}
\newcommand\bet{\rightarrow_\beta}
\newcommand\beq{=_\beta}
\newcommand\Rel{\text{Rel}}
\newcommand\Oper{\text{Oper}}
\newcommand\Inv{\text{Inv}}
\newcommand\Pol{\text{Pol}}
\renewcommand\P{\mathscr{P}}
\newcommand\ar{\text{ar}}

\title{Mémoire de L3~: Un approche catégorique au problème CSP}
\author{}

\begin{document}

\maketitle

\section{Premier rendez vous}

Soit $D$ un ensemble fini.

% TODO définition CSP et logique régulière

\subsection{Adjunction}

\begin{defi}{$\Rel$}
    $\Rel$ est l'ensemble des ensembles de relations sur $D$.

    \[\Rel = \P\left(\bigcup_{n\in\N^*} \P(D^n)\right)\]
\end{defi}

\begin{defi}{$\Oper$}
    $\Oper$ est l'ensemble des ensemble d'opérations sur $D$.

    \[\Oper = \P\left(\bigcup_{n\in\N^*} \P(D^{D^n})\right)\]
\end{defi}

Ces deux ensembles sont partiellement ordonnables par l'inclusion. On va alors
chercher à créer une connection de Galois (ou adjonction/dualité) entre ces
ensembles.

Pour cela, on va définir une relation entre les opérations et les relations sur $D$.

\begin{defi}{$\bot$} Soit $R\in \bigcup_{n\in\N^*} \P(D^n)$ une relation sur $D$
    et $f\in\bigcup_{n\in\N^*} \P(D^{D^n})$ une opération sur $D$. On écrit alors
    $R\bot f$ si, en posant $n = \ar(R)$ et $m = \ar(f)$~:

    \[ \forall (a_{i,j})\in D^{nm}, (\forall 1\leq j\leq m, R(a_{1,j}, \dots a_{n, j}))
         \implies R(f(a_{1,1}, \dots a_{1,m}), \dots f(a_{n,1}, \dots a_{n,m})) \]
\end{defi}

On va maintenant définir les deux foncteurs contrinvariant (fonctions décroissantes)
entre $\Rel$ et $\Oper$ qui formeront notre adjunction.

\begin{defi}{$\Pol$ et $\Inv$}
    On définit~:

    \[\Pol:\begin{array}{rcl}\Rel & \rightarrow & \Oper \\
        S & \mapsto & \{f\in\Oper | \forall R\in S, R\bot f\} \\
    \end{array}\]

    \[\Inv:\begin{array}{rcl}\Oper & \rightarrow & \Rel \\
        S & \mapsto & \{R\in\Rel | \forall f\in S, R\bot f\} \\
    \end{array}\]
\end{defi}

\begin{lem}{Fonctorialité}
    $\Pol$ et $\Inv$ sont des foncteurs contrinvariants, respectivement de $\Rel$ vers
    $\Oper$ et de $\Oper$ vers $\Rel$.
\end{lem}

On a maintenant le premier résultat intéressant, qui motive leur définition~:

\begin{theo}{Adjunction}
    $\Pol\dashv\Inv : \Rel^\text{op} \rightarrow \Oper$
\end{theo}
\begin{proof}
    Soit $S_R\in\Rel$ et $S_f\in\Oper$. On a~:

    \begin{align*}
        S_f\subseteq\Pol(S_R) &\iff \forall f\in S_f, \forall R\in S_R, R\bot f \\
                              &\iff \forall R\in S_R, \forall f\in S_f, R\bot f \\
                              &\iff S_R\subseteq\Inv(S_f)
    \end{align*}
\end{proof}

\subsection{Clone}

\begin{defi}{Clone} Un clone est un ensemble d'opérations (donc un élément de $\Oper$)
    contenant l'identité, et stable par composition, permutation, affaiblissement et
    contraction.
\end{defi}

\begin{defi}{Clone relationel} Un clone relationel est un ensemble de relations
    (donc un élément de $\Rel$) contenant l'égalité et clôt par les constructions
    régulières (quantification existentielle et conjonction).
\end{defi}

\begin{defi}{Clone relationel engendré} Pour $\Gamma\in\Rel$, on note $<\Gamma>$ le plus
    petit clone relationel contenant $\Gamma$ (c'est bien défini car l'intersection d'un
    ensemble quelconque de clones relationels est un clone relationel).
\end{defi}

\subsection{Théorème important}

\begin{lem}{}
    \[\forall \Gamma\in\Rel, \Inv(\Pol(\Gamma)) \text{ est un clone relationel}\]
\end{lem}
\begin{proof}
    Soit $\Gamma\in\Rel$.

    Soit $f$ une opération sur $D$. Comme
    $\forall (a_i), (b_i)\in D^{\ar(f)}, (a_i) = (b_i) 
        \implies f(a_1, \dots a_{\ar(f)}) = f(b_1, \dots b_{\ar(f)})$, on a $\Delta\bot f$
    et donc $\Delta\in \Inv(F)$ pour tout $F\in\Oper$. Donc $\Delta\in\Inv(\Pol(\Gamma))$.

    % TODO
\end{proof}

\end{document}

