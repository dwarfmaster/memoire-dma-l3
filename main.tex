\documentclass[12pt]{article}

\input{inc.tex}

\usepackage{mathrsfs}
\usetikzlibrary{positioning}
\usepackage{tikz-cd}
\newcommand\M{\text{M}(X\uplus X^{-1})}
\newcommand\G{\mathscr{G}(X)}
\renewcommand\F{\mathscr{F}(X)}
\newcommand\V{\mathbb{V}}
\newcommand\E{\mathbb{E}}
\newcommand\D{\mathbb{D}}
\newcommand\bet{\rightarrow_\beta}
\newcommand\beq{=_\beta}
\newcommand\Rel{\text{Rel}}
\newcommand\Set{\text{Set}}
\newcommand\Oper{\text{Oper}}
\newcommand\Inv{\text{Inv}}
\newcommand\Pol{\text{Pol}}
\renewcommand\P{\mathscr{P}}
\newcommand\ar{\text{ar}}
\newcommand\csp{\text{CSP}}
\renewcommand\C{\mathscr{C}}
\newcommand\fset{\text{FinSet}}
\newcommand\im{\text{Im }}
\newcommand\sem[1]{\llbracket {#1} \rrbracket}
\newcommand\ext{\text{Ext}}
\newcommand\fns{\mathscr{F}}
\newcommand\brk[1]{[ {#1} ]}

\title{Mémoire de L3~: Un approche catégorique au problème CSP}
\author{}

\begin{document}

\maketitle


%   ____ ____  ____                    _     _                
%  / ___/ ___||  _ \   _ __  _ __ ___ | |__ | | ___ _ __ ___  
% | |   \___ \| |_) | | '_ \| '__/ _ \| '_ \| |/ _ \ '_ ` _ \ 
% | |___ ___) |  __/  | |_) | | | (_) | |_) | |  __/ | | | | |
%  \____|____/|_|     | .__/|_|  \___/|_.__/|_|\___|_| |_| |_|
%                     |_|                                     
\section{Premier rendez vous}
\input csp.tex

%   ____      _                        _           _    ____ ____  ____  
%  / ___|__ _| |_ ___  __ _  ___  _ __(_) ___ __ _| |  / ___/ ___||  _ \ 
% | |   / _` | __/ _ \/ _` |/ _ \| '__| |/ __/ _` | | | |   \___ \| |_) |
% | |__| (_| | ||  __/ (_| | (_) | |  | | (_| (_| | | | |___ ___) |  __/ 
%  \____\__,_|\__\___|\__, |\___/|_|  |_|\___\__,_|_|  \____|____/|_|    
%                     |___/                                              
\section{Une catégorie pour les CSP}
\input faisceau.tex

\end{document}

