\documentclass[12pt]{article}

\usepackage[utf8]{inputenc}
\usepackage[T1]{fontenc}
\usepackage[francais]{babel}
\usepackage{amsmath}
\usepackage{amssymb}
\usepackage{pgfornament}
\usepackage{amsthm}
\usepackage{stmaryrd} %for \llbracket and \rrbracket
%\usepackage{bbold} %for a nice \mathbb{1}
\usepackage{graphicx}
\usepackage{float}
\usepackage{bbm}
\usepackage{arcs}
\usepackage{calc}
\renewcommand{\mathbb}[1]{\mathbbm{#1}}

\everymath{\displaystyle}

\usepackage[
  leftmargin = 0pt,
  innerleftmargin = 1em,
  innertopmargin = 0pt,
  innerbottommargin = 0pt,
  innerrightmargin = 0pt,
  rightmargin = 0pt,
  linewidth = 3pt,
  topline = false,
  rightline = false,
  bottomline = false,
]{mdframed}

\newcommand{\floor}[1]{\lfloor #1 \rfloor}
\newcommand{\ceil}[1]{\lceil #1 \rceil}
\newcommand{\eps}{\varepsilon} % j'aime pas "eps"
\renewcommand{\epsilon}{\varepsilon}
\renewcommand{\phi}{\varphi}

\newcommand{\dd}{\mathrm{d}}
\newcommand{\der}[2]{\frac{\dd #1}{\dd #2}}
\newcommand{\dern}[3]{\frac{\dd^{#3} #1}{\dd {#2}^{#3}}}
\newcommand{\dpar}[2]{\frac{\partial #1}{\partial #2}}
\newcommand{\dparn}[3]{\frac{\partial^{#3} #1}{\partial {#2}^{#3}}}
\newcommand{\infsum}{\sum^{+\infty}}
\newcommand{\infprod}{\prod^{+\infty}}

\newcommand{\sep}{\begin{center} \pgfornament[width=10cm]{88} \end{center}}
\newcommand{\finish}{\begin{center} \pgfornament[width=5cm]{75} \end{center}}
\DeclareMathOperator{\id}{id}


%\linespread{1.1}
%\setlength{\parskip}{1em plus2mm minus2mm}


%  _____ _
% |_   _| |__   ___  ___  _ __ ___ _ __ ___  ___
%   | | | '_ \ / _ \/ _ \| '__/ _ \ '_ ` _ \/ __|
%   | | | | | |  __/ (_) | | |  __/ | | | | \__ \
%   |_| |_| |_|\___|\___/|_|  \___|_| |_| |_|___/

\newtheoremstyle{defistyle}{}{}{\normalfont}{}{\bfseries}{\ :}{ }{}
\newtheoremstyle{theostyle}{}{}{\normalfont}{}{\bfseries}{}{ }{}
\newtheoremstyle{axiostyle}{}{}{\normalfont}{}{\bfseries}{}{ }{}
\newtheoremstyle{preuvestyle}{}{}{\normalfont}{}{\normalfont}{\ :}{ }{}

\definecolor{emerald}{rgb}{0.31, 0.78, 0.47}
\newmdenv[
  linecolor = emerald!100
]{leftbargreen}

\definecolor{seablue}{rgb}{0.0, 0.412, 0.58}
\newmdenv[
  linecolor = seablue!100
]{leftbarblue}

\definecolor{orange}{rgb}{1.0, 0.6, 0.0}
\newmdenv[
  linecolor = orange!100
]{leftbarorange}

\definecolor{red}{rgb}{0.9, 0.0, 0.0}
\newmdenv[
  linecolor = red!100
]{leftbarred}

\definecolor{violet}{rgb}{0.58, 0.0, 0.83}
\newmdenv[
  linecolor = violet!100
]{leftbarviolet}

\definecolor{brown}{rgb}{0.59, 0.29, 0.0}
\newmdenv[
  linecolor = brown!100
]{leftbarbrown}

\theoremstyle{defistyle}
\newtheorem{definth}{Définition}[section]
\theoremstyle{theostyle}
\newtheorem{theonth}{Th\'eor\`eme}[subsection]
\newtheorem*{propnth}{Propriété}
\newtheorem*{propsnth}{Propriétés}
\newtheorem{proponth}[theonth]{Proposition}
\newtheorem{cornth}[theonth]{Corollaire}
\newtheorem{lemnth}[theonth]{Lemme}
\newtheorem*{exnth}{Exemple}
\newtheorem*{exsnth}{Exemples}
\theoremstyle{preuvestyle}
\newtheorem*{remnth}{Remarque}
\newtheorem*{remsnth}{Remarques}
\newtheorem*{pvnth}{Démonstration}
\theoremstyle{axiostyle}
\newtheorem*{axinth}{Axiome}
\newtheorem*{lointh}{Loi}
\newtheorem*{postnth}{Postulat}

\newenvironment{defi}[1]
  {\begin{leftbarred}\begin{definth} \textbf{#1}\ \\}
  {\end{definth}\end{leftbarred}}
\newenvironment{prop}[1][]
  {\begin{leftbarviolet}\begin{proponth} \textbf{#1}\ \\}
  {\end{proponth}\end{leftbarviolet}}
\newenvironment{props}[1][]
  {\begin{leftbarviolet}\begin{propsnth} \textbf{#1} \begin{enumerate}}
  {\end{enumerate}\end{propsnth}\end{leftbarviolet}}
\newenvironment{propo}[1][]
  {\begin{leftbarviolet}\begin{proponth} \textbf{#1}\  \\}
  {\end{proponth}\end{leftbarviolet}}
\newenvironment{theo}[1]
  {\begin{leftbarviolet}\begin{theonth} \textbf{#1}\ \\}
  {\end{theonth}\end{leftbarviolet}}
\newenvironment{cor}[1][]
  {\begin{leftbarviolet}\begin{cornth} \textbf{#1}\ \\}
  {\end{cornth}\end{leftbarviolet}}
\newenvironment{lem}[1][]
  {\begin{leftbarviolet}\begin{lemnth} \textbf{#1}\ \\}
  {\end{lemnth}\end{leftbarviolet}}
\newenvironment{ex}[1][]
  {\begin{leftbarorange}\begin{exnth} \textbf{#1}\ \\}
  {\end{exnth}\end{leftbarorange}}
\newenvironment{exs}[1][]
  {\begin{leftbarorange}\begin{exsnth} \textbf{#1}\begin{enumerate}}
  {\end{enumerate}\end{exsnth}\end{leftbarorange}}
\newenvironment{rem}
  {\begin{leftbarbrown}\begin{remnth}}
  {\end{remnth}\end{leftbarbrown}}
\newenvironment{rems}
  {\begin{leftbarbrown}\begin{remnth}\begin{itemize}}
  {\end{itemize}\end{remnth}\end{leftbarbrown}}
\newenvironment{pv}
  {\begin{leftbargreen}\begin{pvnth}}
  {\hfill$\square$\end{pvnth}\end{leftbargreen}}
\newenvironment{axi}[1][]
  {\begin{leftbarblue}\begin{axinth} \textbf{#1}\ \\}
  {\end{axinth}\end{leftbarblue}}
\newenvironment{loi}[1][]
  {\begin{leftbarblue}\begin{lointh} \textbf{#1}\ \\}
  {\end{lointh}\end{leftbarblue}}
\newenvironment{post}[1][]
  {\begin{leftbarblue}\begin{postnth} \textbf{#1}\ \\}
  {\end{postnth}\end{leftbarblue}}


%   ____                           _                   _   _
%  / ___| ___ _ __   ___ _ __ __ _| |  _ __ ___   __ _| |_| |__  ___
% | |  _ / _ \ '_ \ / _ \ '__/ _` | | | '_ ` _ \ / _` | __| '_ \/ __|
% | |_| |  __/ | | |  __/ | | (_| | | | | | | | | (_| | |_| | | \__ \
%  \____|\___|_| |_|\___|_|  \__,_|_| |_| |_| |_|\__,_|\__|_| |_|___/


\renewcommand{\le}{\leqslant}
\renewcommand{\ge}{\geqslant}

\newcommand{\R}{\ensuremath{\mathbb{R}}}
\newcommand{\Rb}{\ensuremath{\overline{\mathbb{R}}}}
\newcommand{\N}{\ensuremath{\mathbb{N}}}
\newcommand{\Q}{\ensuremath{\mathbb{Q}}}
\newcommand{\Z}{\ensuremath{\mathbb{Z}}}
\newcommand{\C}{\ensuremath{\mathbb{C}}}
\newcommand{\U}{\ensuremath{\mathbb{U}}}
\newcommand{\F}{\ensuremath{\mathbb{F}}}
\newcommand{\K}{\ensuremath{\mathbb{K}}}

\newcommand{\imp}{\;\Rightarrow\;}
\newcommand{\pmi}{\;\Leftarrow\;}
\newcommand{\eq}{\;\Leftrightarrow\;}
\newcommand{\tq}{\;|\;}

\newcommand{\et}{\text{ et }}
\newcommand{\ou}{\text{ ou }}
\newcommand{\si}{\text{ si }}
\newcommand{\sinon}{\text{ sinon }}

\newcommand{\eint}[1]{\llbracket #1 \rrbracket}

\newcommand{\app}[3][0cm]{\hspace{-#1} \mbox{\raisebox{#2}{$\begin{aligned}#3
\end{aligned}$}} }
%auto app
\newcommand{\aapp}[2][0cm]{\hspace{-#1} \mbox{\raisebox{- \height+2ex}{$\begin{aligned}#2
\end{aligned}$}} }




\usepackage{mathrsfs}
\usetikzlibrary{positioning}
\newcommand\M{\text{M}(X\uplus X^{-1})}
\newcommand\G{\mathscr{G}(X)}
\renewcommand\F{\mathscr{F}(X)}
\newcommand\V{\mathbb{V}}
\newcommand\D{\mathbb{D}}
\newcommand\bet{\rightarrow_\beta}
\newcommand\beq{=_\beta}
\newcommand\Rel{\text{Rel}}
\newcommand\Oper{\text{Oper}}
\newcommand\Inv{\text{Inv}}
\newcommand\Pol{\text{Pol}}
\renewcommand\P{\mathscr{P}}
\newcommand\ar{\text{ar}}
\newcommand\csp{\text{CSP}}
\renewcommand\C{\mathscr{C}}
\newcommand\fset{\text{FinSet}}

\title{Mémoire de L3~: Un approche catégorique au problème CSP}
\author{}

\begin{document}

\maketitle


%   ____ ____  ____                    _     _                
%  / ___/ ___||  _ \   _ __  _ __ ___ | |__ | | ___ _ __ ___  
% | |   \___ \| |_) | | '_ \| '__/ _ \| '_ \| |/ _ \ '_ ` _ \ 
% | |___ ___) |  __/  | |_) | | | (_) | |_) | |  __/ | | | | |
%  \____|____/|_|     | .__/|_|  \___/|_.__/|_|\___|_| |_| |_|
%                     |_|                                     
\section{Premier rendez vous}

Soit $D$ un ensemble fini.

\subsection{CSP}

\begin{defi}{Logique régulière}
    Une proposition de la logique régulière est construite à partir d'un nombre
    fini de relations $R_1, \dots R_n$, sur un langage de termes ne contenant que
    des variables $x, y, z \dots$, de la conjonction, de la quantification existentielle
    et du symbole faux.

    \[\begin{array}{rcl}
        P, Q & := & R_1(x_1, \dots, x_{\ar(R_1)}) \\
             & |  & \vdots                        \\
             & |  & R_n(x_1, \dots, x_{\ar(R_n)}) \\
             & |  & x = y                         \\
             & |  & P \wedge Q                    \\
             & |  & \exists x. P                  \\
             & |  & \bot                          \\
    \end{array}\]
\end{defi}

\begin{defi}{Problème de satisfaction de contraintes}
    Pour un ensemble fini $\mathcal{D}$ de contraintes, on donne une interprétation de
    ces contraintes sur $D$. Cela revient à dire que $D$ est un modèle de la théorie
    vide sur le langage de la logique régulière. L'interprétation des différents 
    connecteurs se font comme en théorie des modèle de premier ordre usuel. 
    $\csp(\mathcal{D})$ est alors le problème, pour une proposition de la logique
    régulière $P$, de déterminer si $D\models P$.
\end{defi}

Par la suite, on assimilera une proposition à $m$ variables libres avec une relation
d'arité $m$.

\begin{lem}
    Pour tout $\mathcal{D}$, $\csp(\mathcal{D})$ est dans $NP$.

    On ne peut pas dire mieux dans le cas général. En effet, $3-SAT$ est un $\csp$
    (facile), donc certains $\csp$ sont $NP$-complets.
\end{lem}

\begin{theo}{Dichotomie}
    Soit $\mathcal{D}$. $\csp(\mathcal{D})$ est soit $NP$-complet, soit dans $P$.
\end{theo}

Ce théorème est très difficile. Pour commencer à l'attauqer, on veut pouvoir comparer les
$\csp$ par difficulté. La comparaison qui va nous intéresser est la réduction LOGSpace.

\begin{defi}{Réduction}
    Soient $\mathcal{D}$ et $\mathcal{F}$ des ensembles finis de relations, éventuellement
    sur des domaines différents. On écrit $\csp(\mathcal{D})\leq\csp(\mathcal{F})$ si il
    existe une fonction calculable en espace logarithmique qui envoie une instance de
    $\csp(\mathcal{D})$ sur une instance de $\csp(\mathcal{F})$ qui soit équisatisfiable.
\end{defi}

On va maintenant considérer des critères plus simples qui permettent de comparer des
$\csp$.

\begin{defi}{pp-réduction}
    $\mathcal{F}$ est dit pp-réductible (primitif positif réductible) à
    $\mathcal{D}$ (sur un même domaine) si chaque relation de $\mathcal{F}$ peut
    s'écrire comme une formule régulière sur $\mathcal{D}$.
\end{defi}

\begin{lem}
    Si $\mathcal{F}$ est pp-réductible à $\mathcal{D}$, alors
    $\csp(\mathcal{F})\leq\csp(\mathcal{D})$.
\end{lem}

\subsection{Adjunction}

\begin{defi}{$\Rel$}
    $\Rel$ est l'ensemble des ensembles de relations sur $D$.

    \[\Rel = \P\left(\bigcup_{n\in\N^*} \P(D^n)\right)\]
\end{defi}

\begin{defi}{$\Oper$}
    $\Oper$ est l'ensemble des ensemble d'opérations sur $D$.

    \[\Oper = \P\left(\bigcup_{n\in\N^*} \P(D^{D^n})\right)\]
\end{defi}

Ces deux ensembles sont partiellement ordonnables par l'inclusion. On va alors
chercher à créer une connection de Galois (ou adjonction/dualité) entre ces
ensembles.

Pour cela, on va définir une relation entre les opérations et les relations sur $D$.

\begin{defi}{$\bot$} Soit $R\in \bigcup_{n\in\N^*} \P(D^n)$ une relation sur $D$
    et $f\in\bigcup_{n\in\N^*} \P(D^{D^n})$ une opération sur $D$. On écrit alors
    $R\bot f$ si, en posant $n = \ar(R)$ et $m = \ar(f)$~:

    \[ \forall (a_{i,j})\in D^{nm}, (\forall 1\leq j\leq m, R(a_{1,j}, \dots a_{n, j}))
         \implies R(f(a_{1,1}, \dots a_{1,m}), \dots f(a_{n,1}, \dots a_{n,m})) \]
\end{defi}

On va maintenant définir les deux foncteurs contrinvariant (fonctions décroissantes)
entre $\Rel$ et $\Oper$ qui formeront notre adjunction.

\begin{defi}{$\Pol$ et $\Inv$}
    On définit~:

    \[\Pol:\begin{array}{rcl}\Rel & \rightarrow & \Oper \\
        S & \mapsto & \{f\in\Oper | \forall R\in S, R\bot f\} \\
    \end{array}\]

    \[\Inv:\begin{array}{rcl}\Oper & \rightarrow & \Rel \\
        S & \mapsto & \{R\in\Rel | \forall f\in S, R\bot f\} \\
    \end{array}\]
\end{defi}

\begin{lem}
    $\Pol$ et $\Inv$ sont des foncteurs contrinvariants, respectivement de $\Rel$ vers
    $\Oper$ et de $\Oper$ vers $\Rel$.
\end{lem}

On a maintenant le premier résultat intéressant, qui motive leur définition~:

\begin{theo}{Adjunction}
    $\Pol\dashv\Inv : \Rel^\text{op} \rightarrow \Oper$
\end{theo}
\begin{proof}
    Soit $S_R\in\Rel$ et $S_f\in\Oper$. On a~:

    \begin{align*}
        S_f\subseteq\Pol(S_R) &\iff \forall f\in S_f, \forall R\in S_R, R\bot f \\
                              &\iff \forall R\in S_R, \forall f\in S_f, R\bot f \\
                              &\iff S_R\subseteq\Inv(S_f)
    \end{align*}
\end{proof}

\subsection{Clone}

\begin{defi}{Clone} Un clone est un ensemble d'opérations (donc un élément de $\Oper$)
    contenant l'identité, et stable par composition, permutation, affaiblissement et
    contraction.
\end{defi}

\begin{defi}{Clone relationel} Un clone relationel est un ensemble de relations
    (donc un élément de $\Rel$) contenant l'égalité et clôt par les constructions
    régulières (quantification existentielle et conjonction).
\end{defi}

\begin{defi}{Clone relationel engendré} Pour $\Gamma\in\Rel$, on note $<\Gamma>$ le plus
    petit clone relationel contenant $\Gamma$ (c'est bien défini car l'intersection d'un
    ensemble quelconque de clones relationels est un clone relationel).
\end{defi}

\subsection{Théorème important}

% TODO hypothèse : l'image de Inv est toujours un clone relationel, et l'image de Pol est
% toujours un clone
\begin{lem}{}
    \[\forall \Gamma\in\Rel, \Inv(\Pol(\Gamma)) \text{ est un clone relationel}\]
\end{lem}
\begin{proof}
    Soit $\Gamma\in\Rel$.

    Soit $f$ une opération sur $D$. Comme
    $\forall (a_i), (b_i)\in D^{\ar(f)}, (a_i) = (b_i) 
        \implies f(a_1, \dots a_{\ar(f)}) = f(b_1, \dots b_{\ar(f)})$, on a $\Delta\bot f$
    et donc $\Delta\in \Inv(F)$ pour tout $F\in\Oper$. Donc $\Delta\in\Inv(\Pol(\Gamma))$.

    De manière immédiate, $\Inv(F)$ est clos par intersection de relations pour tout
    $F\in\Oper$.

    Soit $f\in\Pol(\Gamma)$. Soit $R\in\Inv(\Pol(\Gamma))$, avec $\ar(R)>1$. Posons
    $R'$ la relation définie par $R'(x_1, \dots x_{\ar(R)-1}) :\iff
    \exists x. R(x, x_1, \dots x_{\ar(R)-1})$. Posons $m = \ar(R) - 1$ et $n = \ar(f)$.
    Soient maintenant $(a_{i,j})\in D^{mn}$ tels que
    $\forall 1\leq j\leq n, R'(a_{1,j}, \dots, a_{m,j})$. On prends $x_1, \dots x_n$ les
    témoins de chaque relation. En posant $x = f(x_1, \dots x_n)$, et
    $a_i = f(a_{i,1}, \dots a_{i,n})$, on a $R(x, a_1, \dots a_m)$ et donc
    $R'(a_1, \dots a_m)$, d'où le résultat.
\end{proof}

\begin{theo}{1969}
    \[\forall \Gamma\in\Rel, <\Gamma> = \Inv(\Pol(\Gamma))\]
\end{theo}
\begin{proof}\begin{description}
    \item[$\subseteq$] Par propriété d'adjonction, on a 
        $\Gamma\subseteq\Inv(\Pol(\Gamma))$. Or d'après le lemme précédent, 
        $\Inv(\Pol(\Gamma))$ est un clone relationel. Donc, comme $<\Gamma >$ est le
        plus petit clone relationel contenant $\Gamma$, on a 
        $<\Gamma>\subseteq\Inv(\Pol(\Gamma))$.
    \item[$\supseteq$] Difficile ! Donc admis. % TODO
\end{description}\end{proof}

\begin{cor}{Critère de pp-réductibilité}
    $\mathcal{D}$ est pp-réductible à $\mathcal{F}$ si et seulement si
    $\Pol(\mathcal{F})\subseteq\Pol(\mathcal{D})$.
\end{cor}

\begin{proof}
    $\mathcal{D}$ est pp-réductible à $\mathcal{F}$ si et seulement si
    $\mathcal{D}\subseteq<\mathcal{F}>$ (c'est exactement la définition de
    pp-réductibilité). On a alors~:
    \[\begin{array}{rll}
        \mathcal{D}\subseteq<\mathcal{F}>
          &\iff \mathcal{D}\subseteq\Inv(\Pol(\mathcal{F}))
            & \text{ par le théorème précédent} \\
          &\iff \Pol(\mathcal{F})\subseteq\Pol(\mathcal{D})
            & \text{ par propriété d'adjunction} \\
    \end{array}\]
\end{proof}


%   ____      _                        _           _    ____ ____  ____  
%  / ___|__ _| |_ ___  __ _  ___  _ __(_) ___ __ _| |  / ___/ ___||  _ \ 
% | |   / _` | __/ _ \/ _` |/ _ \| '__| |/ __/ _` | | | |   \___ \| |_) |
% | |__| (_| | ||  __/ (_| | (_) | |  | | (_| (_| | | | |___ ___) |  __/ 
%  \____\__,_|\__\___|\__, |\___/|_|  |_|\___\__,_|_|  \____|____/|_|    
%                     |___/                                              
\section{Une catégorie pour les CSP}

Soit $\Sigma = \{R_1, \dots R_n\}$ un alphabet de symboles de relations, avec
leurs arités $a_1, \dots a_n\in\N$.

L'objectif est de créer une catégorie $\C_\Sigma$, que l'on notera $\C$ en l'absence
d'ambiguïté sur $\Sigma$, telle que la donnée d'un problème CSP soit exactement la donnée
d'un faisseau sur $\C$ (à valeur dans $\fset$, puisqu'on se limite à ce cadre là). Pour
pouvoir parler de faisseau, il faudra munir $\C$ d'une structure de site.

\subsection{Définition de $\C$}

Afin de simplifier, on va essayer de donner une représentation des formules indépendante
du renommage des variables (pour éviter les problèmes d'alpha-équivalence) et qui réduit
la structure d'une formule à son plus simple.

La donnée d'une formule est alors un $n$-uplet $r\in\N^n$, avec $r_1$ qui indique le
nombre d'occurence du prédicat $R_1$, $r_2$ le nombre d'occurences du prédicat $R_2$,
\dots On ne dit pour l'instant rien sur les variables, et on appelle $r$ le 
\emph{squelette} de la formule.

\begin{defi}{Lieu de variable}
    Étant donné $r\in\N^n$ un squelette de formule, on va définir l'ensemble des
    \emph{lieux de variables} de la façon suivante~:

    \[ \V(r):= \{(i,j,k)\in\N^3 |        1\leq i\leq n
                                  \wedge 1\leq j\leq r_j
                                  \wedge 1\leq k\leq a_j\} \]

    Intuitivement, un triplet $(i,j,k)\in\V(r)$ désigne un endroit où une variable peut
    aller dans la formule, plus précisément cela désigne le $k$-ième argument de la
    $j$-ième instance d'un prédicat $R_i$.
\end{defi}

Dans une formule, tous les emplacement possibles ne sont pas nécessairement remplis
par des variables différentes, il faut donc un moyen d'assimiler les lieus de variable
qui sont occupés par la même variable. On fait ça de façon brutale en se munissant d'une
relation d'équivalence sur $\V(r)$.

\begin{defi}{Formule}
    Une \emph{formule} est alors un couple $(r,R)\in\N^n\times\P(\V(r)^2)$ où $R$ est
    une relation d'équivalence. On nomme une \emph{variable} de $(r,R)$ une classe
    d'équivalence de lieus, c'est à dire un élément de $\V(r)/R$.
\end{defi}

La catégorie $\C$ va alors avoir comme objets les formules telles que définit ci-dessus,
plus un objet spécial $\star$.

Il faut maintenant définir une notion de morphisme sur $\C$. On va distinguer les
morphismes entre formule, et les morphismes depuis $\star$, qui a un rôle particulier.
Commençons par les morphismes depuis $\star$. Ces morphismes sont simples~: ils 
représentent les variables de la formules visée.

\begin{defi}{Morphismes depuis $\star$}
    On pose $\C(\star,\star) = \{\text{id}_\star\}$ (cet objet n'a pas d'intérêt autre
    que d'être nécessaire). Plus intéressant, on pose~:

    \[\C(\star, (r,R)) = \V(r)/R \]
\end{defi}

Maintenant, il faut définir les morphismes entre formules. L'idée est qu'un morphisme
d'une formule vers une autre est une instance de la première formule dans la deuxième.
Intuitivement, ça indique que la première formule est une sous-formule de la deuxième,
et de quelle manière. Pour cela, il faut un moyen d'indiquer où vont les anciennes
instances des relations. On fait ça à l'aide d'un \emph{déplacement}.

\begin{defi}{Déplacement}
    Un \emph{déplacement} de $r$ vers $r'$ est un n-uplet de fonctions $\phi$, tel que
    $\forall 1\leq i\leq n, \phi_i : r_i\hookrightarrow r_i'$.

    On note l'ensemble des déplacements de $r$ vers $r'$ $\D(r,r')$
\end{defi}

\begin{rem}
    On remarque que $\D(r,r')\neq\emptyset\implies r\leq r'$ où $\leq$ est l'ordre
    partiel point par point.
\end{rem}

On va maintenant définir l'action d'un déplacement sur une relation.

\begin{defi}{Déplacement d'une relation}
    Soient $(r,R)$ et $(r',R')$ deux formules. Soit $\phi\in\D(r,r')$. On pose~:

    \[ \begin{array}{rccl}
          \phi(R) &:= & & \{ ((i,\phi_i(j), k), (i', \phi_{i'}(j'), k'))
                           ~|~ ((i,j,k), (i',j',k'))\in R \} \\
                  &   & \cup &\{ ((i,j,k), (i,j,k)) ~|~ (i,j,k)\in\V(r') \} \\
    \end{array} \]
\end{defi}

On peut vérifier que cette définition a bien un sens~:

\begin{lem}
    Soient $(r,R)$ et $(r',R')$ deux formules. Soit $\phi\in\D(r,r')$.
    Alors $\phi(R)$ est une relation d'équivalence sur $\V(r')$.
\end{lem}

\begin{proof}
    TODO mais facile.
\end{proof}

On a maintenant tout en main pour définir les morphismes entre formules.

\begin{defi}{Morphismes entre formules}
    Soient $(r,R)$ et $(r',R')$ deux formules. On pose~:

    \[\C((r,R),(r',R')) := \{ \phi\in\D(r,r') ~|~ \phi(R)\subseteq R' \}\]
\end{defi}

\begin{rem}\begin{itemize}
    \item $\C((r,R), (r',R'))\neq\emptyset \implies r\leq r'$, ce qui est cohérent
        avec l'intuition derrière la définition des morphismes.
    \item $\C((r,R), (r,R))$ n'est jamais vide, il contient toujours le $n$-uplet
        des identités des $r_i$. Ce sera d'ailleurs la flèche identité pour la loi
        de composition que l'on définira plus bas.
    \item $\star$ n'a aucune flèche entrante en dehors de son identité.
\end{itemize}\end{rem}

On a maintenant besoin de quelques définitions et lemmes préliminaires pour pouvoir
définir proprement la composition de morphismes dans $\C$.

\begin{lem}
    Soient $(r,R)$, $(r',R')$ et $(r'',R'')$ des formules.
    Soient $\phi_1\in\C((r,R),(r',R'))$ et $\phi_2\in\C((r',R'),(r'',R''))$.
    Alors $\phi_2\circ\phi_1\in\C((r,R),(r'',R''))$ où $\circ$ est compris
    terme à terme.
\end{lem}

\begin{proof}
    Comme la composition de deux injections est un injection, on a que
    $\phi_2\circ\phi_1\in\D(r,r'')$.

    On sait que l'on a $\phi_1(R)\subseteq R'$ et $\phi_2(R')\subseteq R''$. Or
    les déplacements sont immédiatements croissants pour l'inclusion lorsque pensés
    comme fonctions de relations.
    Donc $\phi_2\circ\phi_1(R)\subseteq\phi_2(R')\subseteq R''$.
\end{proof}

\begin{defi}{Déplacement d'un lieu de variable}
    Soient $(r,R)$ et $(r',R')$ des formules, et $\phi\in\D(r,r')$.
    Soit $(i,j,k)\in\V(r)$. On définit $\phi((i,j,k)) = (i, \phi_i(j), k)\in\V(r')$.
\end{defi}

\begin{lem}
    Soient $(r,R)$ et $(r',R')$ des formules, et $\phi\in\C((r,R),(r',R'))$.
    On a que~:
    
    \[\forall v_1,v_2\in\V(r), (v_1,v_2)\in\R\implies (\phi(v_1),\phi(v_2))\in R'\]
\end{lem}

\begin{defi}{Déplacement d'une variable}
    Soient $(r,R)$ et $(r',R')$ des formules, et $\phi\in\C((r,R),(r',R'))$.
    Soit $c\in\V(r)/R$. Soit $v\in c$. On pose $\phi(c) = [\phi(v)]\in\V(r')/R'$.

    Le lemme précédent nous assure que c'est bien définit.
\end{defi}

On a maintenant tout en main pour définir la composition de morphismes.

\begin{defi}{Composition dans $\C$}
    Il y a trois cas à considérer~:\begin{itemize}
        \item $id_\star\in\C(\star,\star)$ et $f\in\C(\star, x)$ où $x$ est n'importe
            quel object de $\C$. Alors $f\circ id_\star := f$.
        \item $c\in\C(\star, (r,R))$ et $\phi\in\C((r,R), (r',R'))$. Alors
            on pose $\phi\circ c := \phi(c)$.
        \item $\phi_1\in\C((r,R),(r',R'))$ et $\phi_2\in\C((r',R'),(r'',R''))$. Alors
            on pose $\phi_2\circ\phi_1$ définit terme à terme, comme dans le lemme
            précédent.
    \end{itemize}
\end{defi}

\begin{lem}
    Pour toute formule $(r,R)$, on a que
        $\phi=(id_{r_1},\dots id_{r_n})\in\C((r,R),(r,R))$.
    De plus, $\forall f\in\C(x, (r,R)), \phi\circ f = f$
    et $\forall \phi'\in\C((r,R), (r',R')), f\circ\phi = f$.
\end{lem}

\begin{lem}
    Soient $(r,R)$, $(r',R')$ et $(r'',R'')$ trois formules. Soit $c\in\C(\star,(r,R))$,
    $\phi_1\in\C((r,R), (r',R'))$ et $\phi_2\in\C((r',R'), (r'',R''))$.

    Alors $(\phi_2\circ\phi_1)\circ c = \phi_2\circ (\phi_1\circ c)$.
\end{lem}

\begin{lem}
    Soient $(r_1,R_1)$, $(r_2,R_2)$, $(r_3,R_3)$ et $(r_4,R_4)$ quatres formules.
    Soient $\phi_1\in\C((r_1,R_1), (r_2,R_2))$, $\phi_2\in\C((r_2,R_2), (r_3,R_3))$ et
    $\phi_3\in\C((r_3,R_3),(r_4,R_4))$.

    Alors $\phi_3\circ(\phi_2\circ\phi_1) = (\phi_3\circ\phi_2)\circ\phi_1$.
\end{lem}

\begin{proof} Les trois lemmes sont immédiats. Il suffit d'écrire la définition de $\circ$
    dans chacun des cas pour que ça marche.
\end{proof}

\begin{theo}{Définition de $\C$}
    $\C$ est une catégorie.
\end{theo}

\begin{proof} Conséquence immédiate des trois lemmes précédents.
\end{proof}
\subsection{Structure de site}

% TODO définition d'un site

% TODO

\subsection{Correction de la définition}

% TODO

\end{document}

