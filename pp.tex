
On définit une catégorie qui va contenir des CSPs sur des alphabets quelconques,
et les morphismes vont représenter la pp-interprétabilité. À cette fin il nous
faut un moyen de porter un faisceau sur $\C_\Sigma$ en un foncteur de $\tsigma$
vers $\ffun$.

\begin{defi}{\ffun}
    On définit $\ffun$ la catégorie $\fset^\mathbb{2}$ où $\mathbb{2}$ est la
    catégorie engendrée par le préordre $(2,<)$. C'est la catégorie dont les
    objets sont des flèches de $\fset$ et les flèches des carrés commutatifs.
\end{defi}

\begin{lem}
    Soit $F\in\shc{\C_\Sigma}$. $F$ envoie les colimites de $\C_\Sigma$ vers
    des limites de $\fset$.
\end{lem}

\begin{pv}
    D'après le théorème \ref{canEq}, on sait que $F\cong (Y\circ L) F$. Or tout préfaisceau
    représentable est exact à gauche (ie préserve les limites de $\C_\Sigma^\op$).
\end{pv}

\begin{defi}{Portage de foncteur}
    Soit $F:\C_\Sigma^\op\rightarrow\fset$. On définit $\tf$ un foncteur de
    $\tsigma$ vers $\ffun$ de la façon suivante.

    On envoie un objet $f:n\ast\rightarrow X$ vers la flèche
    $F f :FX\rightarrow(F\ast)^n$ et $\tf$ envoie chaque côté d'un
    carré commutatif sur son image par $F$.
\end{defi}

\begin{defi}{Catégorie $\fcsp$}
    On définit la catégorie $\fcsp$ pour \emph{fonctorial CSP} de la façon suivante.

    Ses objets sont des couples $(\Sigma, F)$ où $\Sigma$ est un alphabet et
    $F\in\shc(\C_\Sigma)$ est un faisceau.

    Un morphisme de $(\Sigma,F)$ vers $(\Gamma,G)$ est un couple $(r,\theta)$ où $r$
    est un foncteur exact à droite de $\tsigma$ vers $\tgamma$ et $\theta$ est
    une transformation naturelle épimorphique de $\tilde{G}\circ r$ vers $\tilde{F}$.
    On obtient le diagramme suivant~:

    \[\begin{tikzcd}
        \tsigma^\op\arrow[dr, bend right, swap, "r^\op"]\arrow[rr, bend left, "\tf"]
            & \Uparrow\theta
            & \ffun \\
        & \tgamma^\op\arrow[ur, bend right, swap, "\tg"] \\
    \end{tikzcd}\]

    La composition se fait comme on l'imagine, en composant les $r$ et
    $\theta$ avec $r|\theta'$.
\end{defi}

\subsection{Adjunction}

On va maintenant essayer de construire un foncteur $\alpha : \csp\rightarrow\fcsp$
et un foncteur $\gamma : \fcsp\rightarrow\csp$ pour former une adjonction.

\subsubsection{Foncteur $\alpha$}

\begin{defi}{Foncteur $\alpha$}
    On définit $\alpha : \fcsp\rightarrow\csp$.

    Soit $(\Sigma,F)\in\fcsp$. On lui associe le CSP d'ensemble sous-jacent $C=F\ast$.
    De plus, si on écrit $\Sigma$ de la façon suivante~:

    \[\begin{tikzcd}
        & \star \arrow[ddl, shift right, swap, "\pi_1^1"]
                \arrow[ddl]
                \arrow[ddl, shift left, "\pi_{\text{ar}(R_1)}^1"]
                \arrow[ddr, shift right, swap, "\pi_1^m"]
                \arrow[ddr]
                \arrow[ddr, shift left, "\pi_{\text{ar}(R_m)}^m"]
                \\
        & & \\
        R_1 & \dots & R_m \\
    \end{tikzcd}\]

    On définit $\mathcal{C} := \{\sem{R_1}_F,\dots\sem{R_m}\}$ où
    $\sem{R_i}_F = \{(F\pi^i_1(x), \dots, F\pi^i_{\ar(R_i)}) | x\in FyR_i\}$.

    On peut alors poser $\alpha(\Sigma,F) = (C,\mathcal{C})$.

    On doit maintenant définir l'image d'une flèche~:

    \[\begin{tikzcd}
        \tsigma^\op\arrow[dr, bend right, swap, "r^\op"]\arrow[rr, bend left, "\tf"]
            & \Uparrow\theta
            & \ffun \\
        & \tgamma^\op\arrow[ur, bend right, swap, "\tg"] \\
    \end{tikzcd}\]

    On lui associe le triplet $(n,F,f)$ où $n = U_1r\iota\ast$,
    $F = \im \tg r\iota\ast$ et $f = \theta^1_{\ast|F}$.
\end{defi}

\begin{rem}
    Dans les preuves qui vont suivre, on utilisera souvent le fait $\iota$, $\arf$,
    $U_1$ et $U_2$ sont exacts à droite, et donc que si on a
    $r : \tsigma\rightarrow\tgamma$ exact à droite, alors tous les chemins dans le
    diagramme ci-dessous sont exacts à droite.
    
    \[\begin{tikzcd}
        \cf\arrow[dr, "\iota\circ\arf"] & & & \cf \\
        & \tsigma\arrow[r, "r"] & \tgamma\arrow[ur, "U_1"]\arrow[dr, "U_2"] & \\
        \C_\Sigma\arrow[ur, "\iota"] & & & \C_\Sigma \\
    \end{tikzcd}\]

    Donc on pourra raisonner sur $r$ en précomposant et postcomposant par les bon
    foncteurs.
\end{rem}

\begin{prop}
    $\alpha$ est bien définit.
\end{prop}

\begin{pv}
    Il faut vérifier que l'image d'un morphisme est bien un morphisme. Considérons le
    morphisme suivant entre $(\Sigma,F)$ et $(\Gamma,G)$~:

    \[\begin{tikzcd}
        \tsigma^\op\arrow[dr, bend right, swap, "r^\op"]\arrow[rr, bend left, "\tf"]
            & \Uparrow\theta
            & \ffun \\
        & \tgamma^\op\arrow[ur, bend right, swap, "\tg"] \\
    \end{tikzcd}\]

    On note $F(\Sigma,F) = (C,\mathcal{C})$, $F(\Gamma,G) = (D,\mathcal{D})$ et l'image
    du morphisme $(n, F, f)$. On doit commencer par vérifier que $F\subseteq D^n$ et
    que $f : F\rightarrow C$ est surjective. Regardons le carré commutatif
    $\theta_{\iota\star}$ dans $\ffun$~:

    \[\begin{tikzcd}
        C & D^n\arrow[l, "\theta^1_{\iota\ast}"] \\
        C\arrow[u, "\id"]
            & G\Phi\arrow[u, "\tg r\iota\ast"]\arrow[l, "\theta^2_{\iota\ast}"] \\
    \end{tikzcd}\]

    On a alors $F = \im\tg r\iota\ast\subseteq D^n$ et $f = \theta^1_{\ast|F}$ est
    surjective.

    À proprement parler notre foncteur candidat est au moins bien typé. Il faut
    maintenant s'assurer que les conditions de la définition \ref{ppInterDef}
    sont respectées.

    On commence par la remarque suivante immédiate suivante~: un objet est
    pp-définissable dans $(D,\mathcal{D})$ si et seulement si il est dans
    l'image du foncteur $\im\circ\tg$ où $\im:\ffun\rightarrow\fset$ prends
    l'image d'une flèche.

    Cette remarque rends immédiat le fait que $F\in<\mathcal{D}>$.

    Rappelons maintenant que $=_D$ est la formule obtenu par $\tf(2\ast\rightarrow\ast)$.
    On veut donc étudier son image par $r$. À cette fin, on regarde l'image par
    $r$ d'un carré commutatif bien choisi, et on déduit des propriétés en composant par
    $U_1$ et $U_2$, et en se servant du fait que tous ces foncteurs sont exacts à droite.
    On va donc considérer le carré suivant~:

    \[\begin{tikzcd}
        2\ast\arrow[r]\arrow[d] & \ast\arrow[d,"\id"] \\
        \ast\arrow[r, "\id"] & \ast \\
    \end{tikzcd}\quad\xrightarrow{r}\quad\begin{tikzcd}
        2n\ast\arrow[r]\arrow[d] & n\ast\arrow[d] \\
        \Phi\arrow[r,"\id"] & \Phi \\
    \end{tikzcd}\quad\xrightarrow{\tg}\quad\begin{tikzcd}
        D^{2n} & D^n\arrow[l,"\Delta"] \\
        G\Phi\arrow[u] & G\Phi\arrow[l,"\id"]\arrow[u] \\
    \end{tikzcd}\]

    L'image de la flèche de gauche dans le dernier diagramme est donc $\Delta(F)$.

    En considérant l'image par la transformation naturelle $\theta$ de ce carré commutatif,
    on obtient un cube, et si on regarde la bonne face on obtient~:

    \[\begin{tikzcd}
        D^{2n}\arrow[r, "f^2"] & C^2 \\
        G\Phi\arrow[u]\arrow[r] & C\arrow[u, "\Delta"] \\
    \end{tikzcd}\]

    Ce qui nous dit exactement ce que l'on veut ! On a donc bien
    $f^{-1}(=_D)\in <\mathcal{D}>$.

    Il reste à traiter le cas des relations. Soit $R\in\Sigma$ une relation d'arité
    $k = \ar(R)$. On va considérer le diagramme commutatif suivant dans $\C_\Sigma$ (qui
    est une flèche de $\tsigma$)~:

    \[\begin{tikzcd}
        k\ast\arrow[d, "\id"]\arrow[r,"\id"] & k\ast\arrow[d] \\
        k\ast\arrow[r]                       & yR \\
    \end{tikzcd}\quad\xrightarrow{r}\quad\begin{tikzcd}
        kn\ast\arrow[d]\arrow[r,"\id"] & kn\ast\arrow[d] \\
        k\Phi\arrow[r]                 & \Psi \\
    \end{tikzcd}\quad\xrightarrow{\tg}\begin{tikzcd}
        D^{kn} & D^{kn}\arrow[l, "\id"] \\
        (G\Phi)^k\arrow[u] & G\Psi\arrow[l]\arrow[u] \\
    \end{tikzcd}\]

    Regardons maintenant le cube commutatif que l'on obtient avec $\theta$~:

    \[\begin{tikzcd}
        D^{kn}\arrow[rr, dashed, red, "f^k"] & &
            C^k & \\
        & D^{kn}\arrow[ul, red, "\id"]
                \arrow[rr, near start, dashed, "\theta_{k\ast\rightarrow yR}^1"] & &
            C^k\arrow[ul, "\id"] \\
        (G\Phi)^k\arrow[uu]\arrow[rr, near start, dashed, "f"] & &
            C^k\arrow[uu, near end, blue, "\id"] & \\
        & G\Psi\arrow[ul]\arrow[uu, red, very thick]
               \arrow[rr, dashed, blue, "\theta_{k\ast\rightarrow yR}^2"] & &
            FyR\arrow[uu]\arrow[ul, blue] \\
    \end{tikzcd}\]

    Les flèches en pointillés sont surjective, donc suivant le chemin rouge, on obtient
    exactement tous les $k$-uplets de la sémantique de $R$, donc on les obtient aussi
    en suivant le chemin bleu. Donc l'image de la flèche épaisse est bien
    $f^{-1}(\sem{R}_F)$, d'où $f^{-1}(\sem{R}_F)\in<\mathcal{D}>$.
\end{pv}

\begin{prop}
    $\alpha$ est bien un foncteur.
\end{prop}

\begin{pv}
    L'image de l'identité est trivialement l'identité. Il reste à vérifier que $\alpha$
    conserve bien la composition. Soient $(\Sigma,F)$, $(\Gamma,G)$, $(\Lambda,H)$ trois
    objets de $\fcsp$ avec les flèches suivantes~:

    \[\begin{tikzcd}
        \tsigma^\op\arrow[dr, bend right, swap, "r^\op"]\arrow[rr, bend left, "\tf"]
            & \Uparrow\theta
            & \ffun \\
        & \tgamma^\op\arrow[ur, bend right, swap, "\tg"] \\
    \end{tikzcd}\qquad\begin{tikzcd}
        \tgamma^\op\arrow[dr, bend right, swap, "r'^\op"]\arrow[rr, bend left, "\tg"]
            & \Uparrow\theta'
            & \ffun \\
        & \tlambda^\op\arrow[ur, bend right, swap, "\tH"] \\
    \end{tikzcd}\]

    L'image du premier morphisme est le triplet $(n,F,f)$ où $n = U_1r\iota\ast$,
    $F = \im \tg r\iota\ast$ et $f = \theta^1_{\ast|F}$.

    L'image du second morphisme est le triplet $(m,G,g)$ où $m = U_1r'\iota\ast$,
    $G = \im \tH r'\iota\ast$ et $g = \theta'^1_{\ast|F}$.

    L'image de la composition est le triplet $(k,H,h)$ où $k = U_1r'r\iota\ast$,
    $H = \im \tH r'r\iota\ast$ et $h = (\theta\circ(r|\theta'))^1_{\ast|F}$.

    On note $r\iota\ast = n\ast\rightarrow\phi$ et
    $r'\iota\ast = m\ast\rightarrow\psi$. On regarde maintenant le carré commutatif
    suivant~:

    \[\begin{tikzcd}
        n\ast\arrow[d,"\id"]\arrow[r,"\id"] & n\ast\arrow[d] \\
        n\ast\arrow[r] & \phi \\
    \end{tikzcd}\quad\xrightarrow{r'}\quad\begin{tikzcd}
        mn\ast\arrow[r, "\id"]\arrow[d] & mn\ast\arrow[d] \\
        n\psi\arrow[r] & \phi' \\
    \end{tikzcd}\quad\xrightarrow{\tH}\quad\begin{tikzcd}
        E^{mn} & E^{mn}\arrow[l, "\id"] \\
        (H\psi)^n\arrow[u] & H\phi'\arrow[l]\arrow[u] \\
    \end{tikzcd}\]

    En regardant ce que l'on obtient avec les transformations naturelles, et en
    contractant les flèches qui sont l'identité, on obtient le diagramme commutatif
    suivant~:

    \[\begin{tikzcd}
        E^{mn}\arrow[rr, "\theta'^1_{r\iota\ast} = (\theta'^1_{\iota\ast})^n"] & &
            D^n\arrow[dr, "\theta^1_{\iota\ast}"] & \\
        & (H\psi)^n\arrow[ur, swap, "(\theta'^2_{\iota\ast})^n"]\arrow[ul] & &
            C \\
        H\phi'\arrow[ur]\arrow[rr, "\theta'^2_{r\iota\ast}"]\arrow[uu] & &
            G\phi\arrow[ur, "\theta^2_{\iota\ast}"]\arrow[uu] & \\
    \end{tikzcd}\]

    Ce diagramme nous permet de conclure immédiatement.
\end{pv}

\subsubsection{Foncteur $\gamma$}\label{secGamma}

Il reste à définire un foncteur de $\csp$ vers $\fcsp$.

Pour l'image d'un CSP $(C,\mathcal{C})$, on peut lui associer le couple
$(\Sigma, YX)$ où $\Sigma$ est l'aphabet sur lequel est définit $\mathcal{C}$
et $X : \Sigma^\op\rightarrow\fset$ le foncteur qui à $\star$ associe $C$, à
une relation associe sa définition dans $\mathcal{C}$ et à une flèche $f_j$
associe la projection d'un $n$-uplet sur la $j$-ième coordonnée. La proposition
\ref{shSemCorrect2} nous dit que cette sémantique est celle à laquelle on pense
naturellement.

Cependant pour les morphismes c'est beaucoup moins facile~: il faut montrer
que fixer $n$, $f$ et $F$ détermine complètement le $r$ et $\theta$, où au moins
avoir un moyen de les choisir de manière canonique qui soit compatible avec la
composition. On a de bonnes raisons de croire que c'est possible, mais il reste
du travail à faire. Le théorème \ref{lifting} est un pas dans la bonne direction.

Ce n'est pas tout, puisque on aimerait bien que $\alpha$ et $\gamma$ se comportent
bien vis-à-vis l'un de l'autre. Il est probablement trop ambitieux d'obtenir
une équivalence de catégorie, mais avoir une adjonction dans un sens ou dans
l'autre serait intéressant. De plus étudier la pleinitude/fidélité de ces foncteurs
serait probablement intéressant.

