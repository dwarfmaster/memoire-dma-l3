Soit $D$ un ensemble fini.

\begin{defi}{$\Rel$}
    $\Rel$ est l'ensemble des ensembles de relations sur $D$.

    $$\Rel = \P\left(\bigcup_{n\in\mathbb{N^*}} \P(D^n)\right)$$
\end{defi}

\begin{defi}{$\Oper$}
    $\Oper$ est l'ensemble des ensembles d'opérations sur $D$.

    $$\Oper = \P\left(\bigcup_{n\in\mathbb{N^*}} \P(D^{D^n})\right)$$
\end{defi}

\begin{defi}{Relation $\bot$}
    Soit $R$ une relation d'arité $n$ et $f$ une opération d'arité $k$ sur $D$.
    On écrit que $R \bot f$ lorsque pour tout $(a_{i,j}) \in D^{nk}$,
    
    $$ \forall 1 \leq j \leq k, (a_{1,j},\dots,a_{n,j}) \in R \implies
    (f(a_{1,1},\dots,a_{1,k}),\dots,f(a_{n,1},\dots,a_{n,k})) \in R$$
    
    Cela signifie que $f$ est compatible avec la relation $R$.
\end{defi}

On définit:

\begin{defi}{Pol et Inv}
    $$\Pol:\begin{array}{rcl}\Rel & \rightarrow & \Oper \\
    S & \mapsto & \{f\in\Oper | \forall R\in S, R\bot f\} \\
    \end{array}$$
    
    $$\Inv:\begin{array}{rcl}\Oper & \rightarrow & \Rel \\
    S & \mapsto & \{R\in\Rel | \forall f\in S, R\bot f\} \\
    \end{array}$$
\end{defi}

Pour un CSP $(D,\mathcal{D})$, $\Pol(\mathcal{D})$ est l'ensemble des
polymorphismes du CSP. Cet ensemble contient les endomorphismes (opérations
d'arité 1), qui peuvent être vus comme des symétries (car ils préservent les
relations). Les autres éléments de $\Pol(\mathcal{D})$ peuvent être vus comme
des symétries de plus grande arité.

\begin{prop}
    $\Pol(\mathcal{D})$ a les propriétés suivantes :
    \begin{itemize}
    	\item $\Pol(\mathcal{D})$ contient toutes les projections
        \item $\Pol(\mathcal{D})$ est clos par composition, c'est à dire, si
            $g$ est une opération d'arité $k$ et si $f_1,\dots,f_k$ sont des
            opérations d'arité $n$, l'opération $g(f_1,\dots,f_k)$ est définie
            par :

            $$ \forall (a_1,\dots,a_n) \in \mathcal{D}^n,\
            g(f_1,\dots,f_k)(a_1,\dots,a_n) =
            g(f_1(a_1,\dots,a_n),\dots,f_k(a_1,\dots,a_n)) $$
    \end{itemize}
    Les éléments de $\Oper$ qui vérifient ces propriétés sont appelés des
    clones concrets. $\Pol(\mathcal{D})$ est donc le clone (concret) des
    polymorphismes de $\mathcal{D}$.
\end{prop}

\begin{theo}{}
    Soient $(D,\mathcal{D})$ et $(D,\mathcal{E})$ deux CSP sur le domaine $D$.
    $(D,\mathcal{D})$ pp-définit $(D,\mathcal{E})$ si et seulement si
    $\Pol(\mathcal{D}) \subset \Pol(\mathcal{E})$
\end{theo}

\begin{proof}
    L'implication $\implies$ découle de la définition de pp-définissabilité

    L'autre implication se montre comme dans le cas des CSP idempotents : En
    considérant une relation $R$ de $\mathcal{E}$ compatible avec toutes les
    relations de $\Pol(\mathcal{D})$, alors $R$ est pp-définissable dans
    $\mathcal{D}$. Il suffit de constater que l'ensemble des polymorphismes
    $k$-aires de $\Pol(\mathcal{D})$ peut être vu comme une relation
    $|D|^k$-aire S pp-définissable dans $D$. $R$ peut être défini par une telle
    relation en projetant sur des coordonnées adéquates.
\end{proof}

\begin{defi}{}
    Un clone abstrait est un clone dont on ne précise pas le domaine. C'est une
    structure algébrique où seule importe la composition des opérations, et
    quelles opérations sont les projections. C'est le "squelette" d'un clone
    concret.  Soient $\mathbb{D}$ et $\mathbb{E}$ deux clones abstraits. Une
    application $H : \mathbb{D} \rightarrow \mathbb{E}$ est un morphisme de
    clones si :
    \begin{itemize}
    	\item $H$ préserve l'arité des opérations.
    	\item $H$ envoie les projections sur les projections.
        \item $H$ préserve la composition, c'est à dire, pour $f_1,\dots,f_n,g
            \in \mathbb{D}$, avec $g$ d'arité $n \in \mathbb{N}$:
            $$H(g(f_1,\dots,f_n)) = H(g)(H(f_1),\dots,H(f_n)) $$
    \end{itemize}
\end{defi}

\begin{theo}{}
    Soient $(D,\mathcal{D})$ et $(E,\mathcal{E})$ deux CSP.  $(D,\mathcal{D})$
    pp-interprète  $(E,\mathcal{E})$ si et seulement si il existe un morphisme
    de clone de $\Pol(\mathcal{D})$ vers $\Pol(\mathcal{E})$
\end{theo}

Ce résultat montre que seul importe le clone abstrait d'un CSP pour déterminer
sa complexité. On peut ainsi obtenir des conditions sur un CSP
$(E,\mathcal{E})$ pour que celui-ci soit pp-interprétable par un autre CSP
$(D,\mathcal{D})$. S'il existe dans $\Pol(\mathcal{D})$ des opérations
vérifiant certaines équations fonctionnelles, alors il doit exister dans
$\Pol(\mathcal{D})$ des opérations vérifiant les même conditions (c'est l'image
des opérations de $\Pol(\mathcal{D})$ par un morphisme de clone $H$).
Inversement, si il n'y a pas pp-interprétabilité, il existe une condition du
type formulé plus haut qui est vérifiée dans $\Pol(\mathcal{D})$ mais pas dans
$\Pol(\mathcal{E})$. Ces conditions sont appelées les conditions de Mal'tsev, et
permettent de caractériser les CSP.

